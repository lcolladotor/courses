%\VignetteIndexEntry{Exercise 04 from Seminar III: R/Bioconductor}
%\VignetteDepends{}
%\VignetteKeywords{R, Bioconductor}
%\VignettePackage{SIII: R/Bioc}
\documentclass[letterpaper,12pt]{article}

%%%%%%%%%%%%%%%%%%%%%%%% Standard Packages %%%%%%%%%%%%%%%%%%%%%%%%%%%%%%%%%%%%%%%%%%
%\usepackage{epsfig}
%\usepackage{graphicx}
%\usepackage{graphics}
%\usepackage{amssymb}
%\usepackage{amsmath}
%\usepackage{mathrsfs}
%\usepackage{caption}
%\usepackage{comment}
\usepackage{fancyvrb}
\usepackage{fancyhdr}

\usepackage[a4paper]{geometry}
\usepackage{hyperref,graphicx}

%\usepackage[spanish]{babel}
%\selectlanguage{spanish}
%\usepackage[utf8]{inputenc}

%%%%%%%%%%%%%%%%%%%%%% some personal commands %%%%%%%%%%%%%%%%%%%%%%%%%%%%%%%%%%%%%%%%%%%%
\newcommand{\pl}[1]{\texttt{#1}}
\newcommand{\myurlshort}[2]{\href{http://#1}{{\textsf{#2}}}}

%%%%%%%%%%%%%%%%%%%%%% headers and footers %%%%%%%%%%%%%%%%%%%%%%%%%%%%%%%%%%%%%%%%%%%%
\pagestyle{fancy} 
\renewcommand{\footrulewidth}{\headrulewidth}

%%%%%%%%%%%%%%%%%%%%%%%%% bibliography  %%%%%%%%%%%%%%%%%%%%%%%%%%%%%%%%%%%%%%%%%%%%%%%
\bibliographystyle{plainnat}

%%%%%%%%%%%%%%%%%%%%%%%%% sweave options  %%%%%%%%%%%%%%%%%%%%%%%%%%%%%%%%%%%%%%%%%%%%%%%




%%%%%%%%%%%%%%%%%%%%%%% opening %%%%%%%%%%%%%%%%%%%%%%%%%%%%%%%%%%%%
\title{\textbf{Seminar III: \texttt{R}/\texttt{Bioconductor}\\ \small August-December 2009}}
\author{Leonardo Collado Torres\\[1em]Bachelor in Genomic Sciences (LCG),\\ UNAM, Cuernavaca, Mexico\\[1em]\texttt{lcollado@lcg.unam.mx}\\[1em]\url{http://www.lcg.unam.mx/~lcollado/}}

\usepackage{Sweave}
\begin{document}
\maketitle

\medskip
\noindent{\small\textbf{Assistants:} Alejandro Reyes \pl{areyes@lcg.unam.mx}, Jos\'e Reyes \pl{jreyes@lcg.unam.mx} and V\'ictor Moreno \pl{jmoreno@lcg.unam.mx}}

\medskip
\noindent{\small\textbf{Note:} Questions through the \myurlshort{foros.nnb.unam.mx/viewforum.php?f=111}{forum} please. Those who are not from the sixth LCG generation send us an email so we can register you on the forum.}

\medskip
\begin{abstract}
With the following exercises you'll tune your skills with packages such as biomaRt that enable you to download public data sets.
\end{abstract}

\section{biomaRt}
  \begin{enumerate}
  \item Use \pl{biomaRt} to access Uniprot. Get all the proteins from the virus domain. Find out which is the most frequent organism. Then find which is the most frequent EC number.
  \item Follow \url{www.ensembl.org/info/website/tutorials/biomart_worked_ex.pdf} and reproduce it using \pl{biomaRt}. Every time they mention "click result" you need to create a different object with the corresponding information.
  \item Make any plot you want with the previous data. Don't use the base plotting functions. Meaning, use \pl{lattice}, \pl{plotrix}, \ldots
  \end{enumerate}


\end{document}
