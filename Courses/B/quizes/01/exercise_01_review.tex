%\VignetteIndexEntry{Syllabus of Seminar III: R/Bioconductor}
%\VignetteDepends{}
%\VignetteKeywords{R, Bioconductor}
%\VignettePackage{SIII: R/Bioc}
\documentclass[letterpaper,12pt]{article}

%%%%%%%%%%%%%%%%%%%%%%%% Standard Packages %%%%%%%%%%%%%%%%%%%%%%%%%%%%%%%%%%%%%%%%%%
%\usepackage{epsfig}
%\usepackage{graphicx}
%\usepackage{graphics}
%\usepackage{amssymb}
%\usepackage{amsmath}
%\usepackage{mathrsfs}
%\usepackage{caption}
%\usepackage{comment}
\usepackage{fancyvrb}
\usepackage{fancyhdr}

\usepackage[a4paper]{geometry}
\usepackage{hyperref,graphicx}

%\usepackage[spanish]{babel}
%\selectlanguage{spanish}
%\usepackage[utf8]{inputenc}

%%%%%%%%%%%%%%%%%%%%%% some personal commands %%%%%%%%%%%%%%%%%%%%%%%%%%%%%%%%%%%%%%%%%%%%
\newcommand{\pl}[1]{\texttt{#1}}
\newcommand{\myurlshort}[2]{\href{http://#1}{{\textsf{#2}}}}

%%%%%%%%%%%%%%%%%%%%%% headers and footers %%%%%%%%%%%%%%%%%%%%%%%%%%%%%%%%%%%%%%%%%%%%
\pagestyle{fancy} 
\renewcommand{\footrulewidth}{\headrulewidth}

%%%%%%%%%%%%%%%%%%%%%%%%% bibliography  %%%%%%%%%%%%%%%%%%%%%%%%%%%%%%%%%%%%%%%%%%%%%%%
\bibliographystyle{plainnat}

%%%%%%%%%%%%%%%%%%%%%%%%% sweave options  %%%%%%%%%%%%%%%%%%%%%%%%%%%%%%%%%%%%%%%%%%%%%%%




%%%%%%%%%%%%%%%%%%%%%%% opening %%%%%%%%%%%%%%%%%%%%%%%%%%%%%%%%%%%%
\title{\textbf{Seminar III: \texttt{R}/\texttt{Bioconductor}\\ \small August-December 2009}}
\author{Leonardo Collado Torres\\[1em]Bachelor in Genomic Sciences (LCG),\\ UNAM, Cuernavaca, Mexico\\[1em]\texttt{lcollado@lcg.unam.mx}\\[1em]\url{http://www.lcg.unam.mx/~lcollado/index_en.php}}

\usepackage{Sweave}
\begin{document}
\maketitle

\medskip
\noindent{\small\textbf{Assistants:} Alejandro Reyes \pl{areyes@lcg.unam.mx}, Jos\'e Reyes \pl{jreyes@lcg.unam.mx} and V\'ictor Moreno \pl{jmoreno@lcg.unam.mx}}

\medskip
\noindent{\small\textbf{Note:} Questions through the \myurlshort{foros.nnb.unam.mx/viewforum.php?f=111}{forum} please. Those who are not from the sixth LCG generation send us an email so we can register you on the forum.}

\medskip
\begin{abstract}
The following exercises are meant to review how to use \pl{R}, create some basic plots and use the \pl{apply} function family.
\end{abstract}

\section{Review}
  \begin{enumerate}
  \item Why does the following expression show a warning? This is part of what rule?
\begin{Schunk}
\begin{Sinput}
> c(2, 3) + c(4, 5, 7)
\end{Sinput}
\end{Schunk}
  \item For all the prime numbers between 1 and 10, calculate its square root. What is the sum, median and mean?
  \end{enumerate}

\section{Plots}
  \begin{itemize}
  \item Read the following file into \pl{R}: \url{ftp://ftp.ebi.ac.uk/pub/databases/genome_reviews/gr2species_phage.txt}\footnote{Look for the useful function for this case} and make the following plots with your username on the title. Check whether using a log10 scale on the $y$ axis helps.
  \begin{enumerate}
  \item Sort the genome sizes (column 2) and plot them in a line with increasing values.
  \item Plot a histogram with a density line for the same data.
  \item Plot a boxplot for the differences between contigous sorted genomes. Meaning, 2nd smallest - smallest, 3rd smallest - 2nd smallest, etc.\footnote{You might want to use \pl{apropos} searching for diff\ldots}
  \item Make a barplot showing the 10 biggest genomes. Include the names\footnote{They have to be redable} on the $x$ axis and every bar has to have a different color and/or density.\footnote{The \pl{which} function might be useful.}
  \end{enumerate}
  \end{itemize}

\section{\pl{Apply} functions}
  \begin{enumerate}
  \item What is the mean genome size for every type of replicon (column 4)? You have an atomic vector and a factor so use \ldots
  \item Create a matrix \pl{mat} with 10 rows and 10 columns and 100 random uniform values from 1 to 10. Create your own function and apply it to every row so that every row will now sum 1 in your new matrix \pl{mat2}.
  \item Using the same matrix \pl{mat}, make the matrix \pl{mat3} with row sums equal to 1 using built in \pl{R} matrix functions. \pl{mat2} and \pl{mat3} should be the same.
  \end{enumerate}

\end{document}
