%\VignetteIndexEntry{Exercise 02 from Seminar III: R/Bioconductor}
%\VignetteDepends{}
%\VignetteKeywords{R, Bioconductor}
%\VignettePackage{SIII: R/Bioc}
\documentclass[letterpaper,12pt]{article}

%%%%%%%%%%%%%%%%%%%%%%%% Standard Packages %%%%%%%%%%%%%%%%%%%%%%%%%%%%%%%%%%%%%%%%%%
\usepackage{fancyvrb}
\usepackage{fancyhdr}

\usepackage[a4paper]{geometry}
\usepackage{hyperref,graphicx}

%\usepackage[spanish]{babel}
%\selectlanguage{spanish}
%\usepackage[utf8]{inputenc}

%%%%%%%%%%%%%%%%%%%%%% some personal commands %%%%%%%%%%%%%%%%%%%%%%%%%%%%%%%%%%%%%%%%%%%%
\newcommand{\pl}[1]{\texttt{#1}}
\newcommand{\myurlshort}[2]{\href{http://#1}{{\textsf{#2}}}}

%%%%%%%%%%%%%%%%%%%%%% headers and footers %%%%%%%%%%%%%%%%%%%%%%%%%%%%%%%%%%%%%%%%%%%%
\pagestyle{fancy} 
\renewcommand{\footrulewidth}{\headrulewidth}

%%%%%%%%%%%%%%%%%%%%%%%%% bibliography  %%%%%%%%%%%%%%%%%%%%%%%%%%%%%%%%%%%%%%%%%%%%%%%
\bibliographystyle{plainnat}

%%%%%%%%%%%%%%%%%%%%%%%%% sweave options  %%%%%%%%%%%%%%%%%%%%%%%%%%%%%%%%%%%%%%%%%%%%%%%




%%%%%%%%%%%%%%%%%%%%%%% opening %%%%%%%%%%%%%%%%%%%%%%%%%%%%%%%%%%%%
\title{\textbf{Seminar III: \texttt{R}/\texttt{Bioconductor}\\ \small August-December 2009}}
\author{Leonardo Collado Torres\\[1em]Bachelor in Genomic Sciences (LCG),\\ UNAM, Cuernavaca, Mexico\\[1em]\texttt{lcollado@lcg.unam.mx}\\[1em]\url{http://www.lcg.unam.mx/~lcollado/}}

\usepackage{Sweave}
\begin{document}
\maketitle

\medskip
\noindent{\small\textbf{Assistants:} Alejandro Reyes \pl{areyes@lcg.unam.mx}, Jos\'e Reyes \pl{jreyes@lcg.unam.mx} and V\'ictor Moreno \pl{jmoreno@lcg.unam.mx}}

\medskip
\noindent{\small\textbf{Note:} Questions through the \myurlshort{foros.nnb.unam.mx/viewforum.php?f=111}{forum} please. Those who are not from the sixth LCG generation send us an email so we can register you on the forum.}

\medskip
\begin{abstract}
Expected solutions for the second set of exercises. Note that your template \pl{.Rnw} file should just be a much simpler file than the one used to make this pdf file: \pl{answer\_02\_sweave.Rnw}
\end{abstract}

\section{Sweave}
  \begin{enumerate}
  \item Create your own template Sweave document.
  \begin{itemize}
  \item Title: course name, homework number
  \item Author: name, email, include a link to your personal academic webpage if you have one. \footnote{You will probably make one this semester on the PHP course.}
  \item Abstract: short description on the homework and any notes you might want to add
  \item A sample homework solution: meaning a short description and some code. For example, how to sum $2 + 3$.
\begin{Schunk}
\begin{Sinput}
> 2 + 3
\end{Sinput}
\begin{Soutput}
[1] 5
\end{Soutput}
\end{Schunk}
  \end{itemize}
  \item For a proper template check: \url{http://www.lcg.unam.mx/~lcollado/B/quizzes/02_answer/answer_02_template.Rnw} which generates \url{http://www.lcg.unam.mx/~lcollado/B/quizzes/02_answer/answer_02_template.pdf}
  \end{enumerate}

\section{ALL dataset}
  \begin{itemize}
  \item You'll have to explore the ALL dataset\footnote{John Quackenbush mentioned it on Monday as the most studied dataset.} and create your first homework as a vignette document.
  \item Install the \pl{ALL} package and explore the \pl{ALL} object.
\begin{Schunk}
\begin{Sinput}
> library(ALL)
> data(ALL)
> ALL
\end{Sinput}
\begin{Soutput}
ExpressionSet (storageMode: lockedEnvironment)
assayData: 12625 features, 128 samples 
  element names: exprs 
phenoData
  sampleNames: 01005, 01010, ..., LAL4  (128 total)
  varLabels and varMetadata description:
    cod:  Patient ID
    diagnosis:  Date of diagnosis
    ...: ...
    date last seen:  date patient was last seen
    (21 total)
featureData
  featureNames: 1000_at, 1001_at, ..., AFFX-YEL024w/RIP1_at  (12625 total)
  fvarLabels and fvarMetadata description: none
experimentData: use 'experimentData(object)'
  pubMedIds: 14684422 16243790 
Annotation: hgu95av2 
\end{Soutput}
\begin{Sinput}
> varLabels(ALL)
\end{Sinput}
\begin{Soutput}
 [1] "cod"            "diagnosis"      "sex"            "age"           
 [5] "BT"             "remission"      "CR"             "date.cr"       
 [9] "t(4;11)"        "t(9;22)"        "cyto.normal"    "citog"         
[13] "mol.biol"       "fusion protein" "mdr"            "kinet"         
[17] "ccr"            "relapse"        "transplant"     "f.u"           
[21] "date last seen"
\end{Soutput}
\end{Schunk}
  \item Select the samples from the B-cell tumors.\footnote{What's the name of the function to look for text in Unix? Well, a function with the same name is available in \pl{R}. Use it}
\begin{Schunk}
\begin{Sinput}
> bcell <- grep("^B", as.character(ALL$BT))
\end{Sinput}
\end{Schunk}
  \item Select those of molecular type \pl{BCR/ABL} or \pl{NEG}.\footnote{A binary operator such as \%in\% is useful here}
\begin{Schunk}
\begin{Sinput}
> subset.mol.biol <- ALL$mol.biol %in% c("BCR/ABL", "NEG")
\end{Sinput}
\end{Schunk}
  \item Combine the previous two subsets and keep the \emph{intersect}ion
\begin{Schunk}
\begin{Sinput}
> subset.pos <- intersect(bcell, which(subset.mol.biol == TRUE))
> ALL.work <- ALL[, subset.pos]
\end{Sinput}
\end{Schunk}
  \item Eliminate unused factor levels on your resulting subset.
\begin{Schunk}
\begin{Sinput}
> ALL.work$BT <- factor(ALL.work$BT)
> ALL.work$mol.biol <- factor(ALL.work$mol.biol)
\end{Sinput}
\end{Schunk}
  \item Use the \pl{nsFilter} function from the \pl{genefilter} package to keep those with \emph{entrez} ID, \emph{GOBP}, remove duplicate \emph{entrez} and the following arguments:
\begin{Schunk}
\begin{Sinput}
> library(genefilter)
> library(hgu95av2.db)
> filtered <- nsFilter(ALL.work, var.fun = IQR, var.cutoff = 0.5, 
+     feature.exclude = "^AFFX", require.entrez = TRUE, require.GOBP = TRUE, 
+     remove.dupEntrez = TRUE)
\end{Sinput}
\end{Schunk}
  \item Meaning that we'll use the interquantile range with a variance cutoff of 0.5 to eliminate those with small variation and by excluding \pl{AFFX} we'll take out the controls AFFY probes.
  \item How many:
  \begin{enumerate}
  \item duplicates were removed?
\begin{Schunk}
\begin{Sinput}
> filtered$filter.log$numDupsRemoved
\end{Sinput}
\begin{Soutput}
[1] 2653
\end{Soutput}
\end{Schunk}
  \item control features were excluded?
\begin{Schunk}
\begin{Sinput}
> filtered$filter.log$feature.exclude
\end{Sinput}
\begin{Soutput}
[1] 19
\end{Soutput}
\end{Schunk}
  \item had low variance (small variation)?
\begin{Schunk}
\begin{Sinput}
> filtered$filter.log$numLowVar
\end{Sinput}
\begin{Soutput}
[1] 3873
\end{Soutput}
\end{Schunk}
  \item had no GO?
\begin{Schunk}
\begin{Sinput}
> filtered$filter.log$numNoGO.BP
\end{Sinput}
\begin{Soutput}
[1] 1528
\end{Soutput}
\end{Schunk}
  \item had no entrez ID?
\begin{Schunk}
\begin{Sinput}
> filtered$filter.log$numRemoved.ENTREZID
\end{Sinput}
\begin{Soutput}
[1] 679
\end{Soutput}
\end{Schunk}
  \end{enumerate}
  \end{itemize}

\end{document}
