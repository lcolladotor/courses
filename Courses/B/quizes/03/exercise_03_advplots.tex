%\VignetteIndexEntry{Exercise 02 from Seminar III: R/Bioconductor}
%\VignetteDepends{}
%\VignetteKeywords{R, Bioconductor}
%\VignettePackage{SIII: R/Bioc}
\documentclass[letterpaper,12pt]{article}

%%%%%%%%%%%%%%%%%%%%%%%% Standard Packages %%%%%%%%%%%%%%%%%%%%%%%%%%%%%%%%%%%%%%%%%%
%\usepackage{epsfig}
%\usepackage{graphicx}
%\usepackage{graphics}
%\usepackage{amssymb}
%\usepackage{amsmath}
%\usepackage{mathrsfs}
%\usepackage{caption}
%\usepackage{comment}
\usepackage{fancyvrb}
\usepackage{fancyhdr}

\usepackage[a4paper]{geometry}
\usepackage{hyperref,graphicx}

%\usepackage[spanish]{babel}
%\selectlanguage{spanish}
%\usepackage[utf8]{inputenc}

%%%%%%%%%%%%%%%%%%%%%% some personal commands %%%%%%%%%%%%%%%%%%%%%%%%%%%%%%%%%%%%%%%%%%%%
\newcommand{\pl}[1]{\texttt{#1}}
\newcommand{\myurlshort}[2]{\href{http://#1}{{\textsf{#2}}}}

%%%%%%%%%%%%%%%%%%%%%% headers and footers %%%%%%%%%%%%%%%%%%%%%%%%%%%%%%%%%%%%%%%%%%%%
\pagestyle{fancy} 
\renewcommand{\footrulewidth}{\headrulewidth}

%%%%%%%%%%%%%%%%%%%%%%%%% bibliography  %%%%%%%%%%%%%%%%%%%%%%%%%%%%%%%%%%%%%%%%%%%%%%%
\bibliographystyle{plainnat}

%%%%%%%%%%%%%%%%%%%%%%%%% sweave options  %%%%%%%%%%%%%%%%%%%%%%%%%%%%%%%%%%%%%%%%%%%%%%%




%%%%%%%%%%%%%%%%%%%%%%% opening %%%%%%%%%%%%%%%%%%%%%%%%%%%%%%%%%%%%
\title{\textbf{Seminar III: \texttt{R}/\texttt{Bioconductor}\\ \small August-December 2009}}
\author{Leonardo Collado Torres\\[1em]Bachelor in Genomic Sciences (LCG),\\ UNAM, Cuernavaca, Mexico\\[1em]\texttt{lcollado@lcg.unam.mx}\\[1em]\url{http://www.lcg.unam.mx/~lcollado/}}

\usepackage{Sweave}
\begin{document}
\maketitle

\medskip
\noindent{\small\textbf{Assistants:} Alejandro Reyes \pl{areyes@lcg.unam.mx}, Jos\'e Reyes \pl{jreyes@lcg.unam.mx} and V\'ictor Moreno \pl{jmoreno@lcg.unam.mx}}

\medskip
\noindent{\small\textbf{Note:} Questions through the \myurlshort{foros.nnb.unam.mx/viewforum.php?f=111}{forum} please. Those who are not from the sixth LCG generation send us an email so we can register you on the forum.}

\medskip
\begin{abstract}
With the following exercises you'll practice creating some advanced plots. You'll have to explain every plot.
\end{abstract}

\section{lattice}
  \begin{enumerate}
  \item Using our object \pl{t1}, create a histogram of the BAC sizes with one panel per chromosome.
  \item With the object \pl{t2} from class, first normalize the position variable per chromosome (use tapply to find out the max values per chromosome). Then create density plots for your normalized position variable. Every chromosome has to have its own panel.
  \item For chromsomes X and Y, and using the normalized position variable, make a densityplot grouping the information by the reference allele. For each chromosome, separate the data by the AK1.allele variable\footnote{Remember that you can use more than 1 factor}. Your resulting plot should have 8 panels.
  \item (Optional) Check out the \pl{latticeExtra} package and make a plot with one of its functions.
  \end{enumerate}

\section{plotrix}
  \begin{enumerate}
  \item Using the original \pl{t2} object, plot for every chromosome the mean position with error bars. We did something very similar with \pl{t1} on class.
  \item Create a bar plot with the table information for the following data:
\begin{Schunk}
\begin{Sinput}
> df <- data.frame(G1 = c(25, 5, 20), G2 = c(30, 6, 22), G3 = c(40, 
+     6, 18))
> df
\end{Sinput}
\begin{Soutput}
  G1 G2 G3
1 25 30 40
2  5  6  6
3 20 22 18
\end{Soutput}
\end{Schunk}
  \item (Optional) With whichever data you want, create an interesting plot using hierobarp. Don't use the default examples. 
  \end{enumerate}


\end{document}
