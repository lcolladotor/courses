%%%%%%%%%%%%%%%%%%%%%%%%%%%%%%%%%%%%%%%%%%%%%%%%%%%%%%%%%%%%%%%%%%%%%%%%%%%
%%%%%%%%%%%%%%%%%%%%%%%%%%%%%%%%%%%%%%%%%%%%%%%%%%%%%%%%%%%%%%%%%%%%%%%%%%%
%%
%% The basic tex file for the header of all the lectures. 
%%

\documentclass{beamer}
\usetheme[hideallsubsections]{Berkeley}

\usepackage{color}
\usepackage{amsfonts}

\definecolor{myblue}{rgb}{0.25, 0, 0.75}
\definecolor{mygold}{rgb}{1,0.8,0.2}
\definecolor{gray}{rgb}{0.5, 0.5, 0.5}
\definecolor{lucia}{rgb}{0.8,0.4,0.7} 

\newcommand{\myurl}[1]{\href{http://#1}{\textcolor{gray}{\texttt{#1}}}}
\newcommand{\myem}[1]{\structure{#1}}
\newcommand{\RPack}[1]{\textcolor{gray}{\textsf{#1}}}
\newcommand{\pl}[1]{\texttt{#1}}
\newcommand{\Rcode}[1]{\texttt{#1}}
\newcommand{\Rfunction}[1]{\href{http://www.statmethods.net/search/index.asp?QU=#1&search=Search&Action=Search}{\textcolor{orange}{\textsf{#1}}}}
\newcommand{\myurlshort}[2]{\href{http://#1}{\textcolor{gray}{\textsf{#2}}}}
\newcommand{\RClass}[1]{\textcolor{mygold}{\textsf{#1}}}
\newcommand{\BIOCfunction}[1]{\textcolor{orange}{#1}}

\setbeamercolor{example text}{fg=lucia}
\setbeamertemplate{sections/subsections in toc}[ball unumbered]
\setbeamertemplate{frametitle continuation}[from second][]
\setbeamertemplate{itemize subitem}[triangle]
\setbeamertemplate{footline}[page number]
\setbeamertemplate{caption}[numbered]

\renewcommand{\footnotesize}{\fontsize{6.10}{12}\selectfont}

\def\argmax{\operatornamewithlimits{arg\,max}}
\def\argmin{\operatornamewithlimits{arg\,min}}


%%%%%%%%%%%%%%%%%%%%%%%%%%%%%%%%%%%%%%%%%%%%%%%%%%%%%%%%%%%%%%%%%%%%%%%%%%%
\title{R / Bioconductor: Curso Intensivo}

\author[]{\myem{Leonardo Collado Torres}\\
  Licenciatura en Ciencias Gen�micas, UNAM\\
  \myurl{www.lcg.unam.mx/\string~lcollado/index.php}\\
}

\date{
  Cuernavaca, M�xico\\
  Oct-Nov, 2008
}






\usepackage{Sweave}
\begin{document}

%%% set up some options for Sweave and R %%%

%%%%%%%%%%%%%%%%%%%%%%%%%%%%%%%%%%%%%%%%%%%%%%%%%%%%%%%%%%%%%%%%%%%%%%%%%%%
%%%%%%%%%%%%%%%%%%%%%%%%%%%%%%%%%%%%%%%%%%%%%%%%%%%%%%%%%%%%%%%%%%%%%%%%%%%
\begin{frame}[allowframebreaks]
  \titlepage
\end{frame}

\begin{frame}[allowframebreaks]
  \frametitle{Título de la presentación}
  \tableofcontents
\end{frame}

%%%%%%%%%%%%%%%%%%%%%%%%%%%%%%%%%%%%%%%%%%%%%%%%%%%%%%%%%%%%%%%%%%%%%%%%%%%
\section{Título de la sección}

\begin{frame}[allowframebreaks]
  \frametitle{Título diapositiva I}
  \begin{itemize}
    \item \textquestiondown Por llenar
  \end{itemize}
\end{frame}

\begin{frame}[allowframebreaks, fragile]
  \frametitle{Título diapositiva II}
  \begin{itemize}
    \item Por llenar
  \end{itemize}
\begin{Schunk}
\begin{Sinput}
> 2 + 4
\end{Sinput}
\begin{Soutput}
[1] 6
\end{Soutput}
\end{Schunk}

\end{frame}
 
%%%%%%%%%%%%%%%%%%%%%%%%%%%%%%%%%%%%%%%%%%%%%%%%%%%%%%%%%%%%%%%%%%%%%%%%%%%

\end{document}

