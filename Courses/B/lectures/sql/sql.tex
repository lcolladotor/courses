\documentclass{beamer}
\usetheme{Montpellier}

\usepackage{color}
\usepackage{amsfonts}
\usepackage{comment}

%%% Al parecer necesito esto en ubuntu para los acentos
%\usepackage[spanish]{babel}
%\selectlanguage{spanish}
%\usepackage[utf8]{inputenc}

\definecolor{myblue}{rgb}{0.25, 0, 0.75}
\definecolor{mygold}{rgb}{1,0.8,0.2}
\definecolor{gray}{rgb}{0.5, 0.5, 0.5}
\definecolor{lucia}{rgb}{0.8,0.4,0.7} 

\newcommand{\myurl}[1]{\href{http://#1}{\textcolor{gray}{\texttt{#1}}}}
\newcommand{\myem}[1]{\structure{#1}}
\newcommand{\myurlshort}[2]{\href{http://#1}{\textcolor{gray}{\textsf{#2}}}}

\newcommand{\RPackage}[1]{\textcolor{gray}{\textsf{#1}}}
\newcommand{\pl}[1]{\texttt{#1}}
\newcommand{\RCode}[1]{\texttt{#1}}
\newcommand{\RFunction}[1]{\textsf{#1}}
\newcommand{\RClass}[1]{\textcolor{mygold}{\textsf{#1}}}
\newcommand{\BIOCfunction}[1]{\textcolor{orange}{#1}}

\setbeamercolor{example text}{fg=lucia}
\setbeamertemplate{sections/subsections in toc}[ball unumbered]
\setbeamertemplate{frametitle continuation}[from second][]
\setbeamertemplate{itemize subitem}[triangle]
\setbeamertemplate{footline}[page number]
\setbeamertemplate{caption}[numbered]
\setbeamertemplate{navigation symbols}{}

\renewcommand{\footnotesize}{\fontsize{6.10}{12}\selectfont}

\def\argmax{\operatornamewithlimits{arg\,max}}
\def\argmin{\operatornamewithlimits{arg\,min}}

%%\bibliographystyle{plain}


\title{Seminar III: R/Bioconductor}
\author{Leonardo Collado Torres \\ lcollado@lcg.unam.mx \\  Bachelor in Genomic Sciences \\ \myurl{www.lcg.unam.mx/\string~lcollado/}}
\date{
August - December, 2009
}








\usepackage{Sweave}
\begin{document}

\begin{frame}[allowframebreaks]
  \titlepage
\end{frame}

\section*{Class outline}

\begin{frame}[allowframebreaks]
  \frametitle{Public Data}
  \tableofcontents[hideallsubsections]
\end{frame}

%%%%%%%%%%%%%%%%%%%%%%%%%%%%%%%%%%%%%%%%%%%%%%%%%%%%%%%%%%%%%%%%%%%%%%%%%%%
\section{Intro}

\begin{frame}[allowframebreaks, fragile]
  \frametitle{Install}
  \begin{itemize}
  \item Packages:
\begin{Schunk}
\begin{Sinput}
> install.packages("RMySQL")
> source("http://bioconductor.org/biocLite.R")
> biocLite("AnnotationDBI")
\end{Sinput}
\end{Schunk}
  \end{itemize}
\end{frame}

%%%%%%%%%%%%%%%%%%%%%%%%%%%%%%%%%%%%%%%%%%%%%%%%%%%%%%%%%%%%%%%%%%%%%%%%%%%
\section{RMySQL}

\begin{frame}[allowframebreaks]
  \frametitle{RMySQL}
  \begin{itemize}
  \item Installing this package is not that simple. 
  \item In Windows, you will need to download a libmysql.dll file and copy it to the RMySQL folder.
  \item On Linux, you need to install the mysql libraries\footnote{Probably the dev ones too for compiling}.
  \item For now use Montealban :)
  \end{itemize}
\end{frame}

\begin{frame}[allowframebreaks, fragile]
  \frametitle{Explore the pkg}
  \begin{itemize}
  \item You can explore quite a bit of the package using
\begin{Schunk}
\begin{Sinput}
> help(package = "RMySQL")
\end{Sinput}
\end{Schunk}
  \item Its \alert{important} to close every connection you open.
  \end{itemize}
\end{frame}

\begin{frame}[allowframebreaks, fragile]
  \frametitle{Sample session}
  \begin{itemize}
  \item A sample session would look be similar to this. First we connect to the database.
\begin{Schunk}
\begin{Sinput}
> con <- dbConnect(MySQL(), user = "lcollado", 
+     password = "LOL", dbname = "BPdB", 
+     host = "kabah.lcg.unam.mx")
\end{Sinput}
\end{Schunk}
  \item Then we do some queries, list the tables, download some tables into data frames, etc.
\begin{Schunk}
\begin{Sinput}
> dbListTables(con)
> df <- dbReadTable(con, "tablename")
\end{Sinput}
\end{Schunk}
  \item And we end by closing the connection:
\begin{Schunk}
\begin{Sinput}
> dbDisconnect(con)
\end{Sinput}
\end{Schunk}
  \end{itemize}
\end{frame}

\begin{frame}[allowframebreaks]
  \frametitle{Queries vs downloading tables}
  \begin{itemize}
  \item SQL is faster for doing joins between tables, so you might want to use queries then.
  \item If all the info is on one table, and you are in local network with the database server, you might prefer to download the table. Then use a tapply or other functions if you have a grouping variable.
  \end{itemize}
\end{frame}

\begin{frame}[allowframebreaks]
  \frametitle{Exercise I}
  \begin{itemize}
  \item Use \BIOCfunction{RMySQL} to access your own database from the Bioinformatics and Statistics I course.
  \item Use a query to retrieve data and make a plot :)
  \item If you don't have your own database, let me know.
  \end{itemize}
\end{frame}

\begin{frame}[allowframebreaks]
  \frametitle{SQLite}
  \begin{itemize}
  \item An alternative for MySQL that seems to be faster is SQLite.
  \item If you are using that SQL language, you might want to install the \pl{RSQLite} R package :)
  \end{itemize}
\end{frame}

%%%%%%%%%%%%%%%%%%%%%%%%%%%%%%%%%%%%%%%%%%%%%%%%%%%%%%%%%%%%%%%%%%%%%%%%%%%
\section{AnnotationDbi}

\begin{frame}[allowframebreaks]
  \frametitle{Intro}
  \begin{itemize}
  \item Its an alive package :) Meaning that its under intensive development.\footnote{Regretabbly, the latest version has an installation bug.} 
  \item This package is the key interface to access the gamma of annotation packages.
  \item It allows you to retrieve data from them using some \pl{R} functions, or by directly using SQL queries.
  \end{itemize}
\end{frame}


\begin{frame}[allowframebreaks]
  \frametitle{A lab}
  \url{http://bioconductor.org/workshops/2009/BioC2009/labs/annotations/AnnotationDbi.pdf}
  \begin{itemize}
  \item Check the above lab :) Its a very complete description of the \pl{AnnotationDbi} package.
  \item Read the document and pay attention to exercises 1, 2, 3, and 5. If you have a working installation, then try to do them :)
  \end{itemize}
\end{frame}


\begin{frame}[allowframebreaks, fragile]
  \frametitle{SessionInfo} \scriptsize
\begin{Schunk}
\begin{Sinput}
> sessionInfo()
\end{Sinput}
\begin{Soutput}
R version 2.10.0 Under development (unstable) (2009-07-25 r48998) 
i686-pc-linux-gnu 

locale:
 [1] LC_CTYPE=en_US.UTF-8      
 [2] LC_NUMERIC=C              
 [3] LC_TIME=en_US.UTF-8       
 [4] LC_COLLATE=en_US.UTF-8    
 [5] LC_MONETARY=C             
 [6] LC_MESSAGES=en_US.UTF-8   
 [7] LC_PAPER=en_US.UTF-8      
 [8] LC_NAME=C                 
 [9] LC_ADDRESS=C              
[10] LC_TELEPHONE=C            
[11] LC_MEASUREMENT=en_US.UTF-8
[12] LC_IDENTIFICATION=C       

attached base packages:
[1] stats     graphics  grDevices
[4] utils     datasets  methods  
[7] base     
\end{Soutput}
\end{Schunk}
\end{frame}


\end{document}

