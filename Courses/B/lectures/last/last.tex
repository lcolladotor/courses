\documentclass{beamer}
\usetheme{Montpellier}

\usepackage{color}
\usepackage{amsfonts}
\usepackage{comment}

%%% Al parecer necesito esto en ubuntu para los acentos
%\usepackage[spanish]{babel}
%\selectlanguage{spanish}
%\usepackage[utf8]{inputenc}

\definecolor{myblue}{rgb}{0.25, 0, 0.75}
\definecolor{mygold}{rgb}{1,0.8,0.2}
\definecolor{gray}{rgb}{0.5, 0.5, 0.5}
\definecolor{lucia}{rgb}{0.8,0.4,0.7} 

\newcommand{\myurl}[1]{\href{http://#1}{\textcolor{gray}{\texttt{#1}}}}
\newcommand{\myem}[1]{\structure{#1}}
\newcommand{\myurlshort}[2]{\href{http://#1}{\textcolor{gray}{\textsf{#2}}}}

\newcommand{\RPackage}[1]{\textcolor{gray}{\textsf{#1}}}
\newcommand{\pl}[1]{\texttt{#1}}
\newcommand{\RCode}[1]{\texttt{#1}}
\newcommand{\RFunction}[1]{\textsf{#1}}
\newcommand{\RClass}[1]{\textcolor{mygold}{\textsf{#1}}}
\newcommand{\BIOCfunction}[1]{\textcolor{orange}{#1}}

\setbeamercolor{example text}{fg=lucia}
\setbeamertemplate{sections/subsections in toc}[ball unumbered]
\setbeamertemplate{frametitle continuation}[from second][]
\setbeamertemplate{itemize subitem}[triangle]
\setbeamertemplate{footline}[page number]
\setbeamertemplate{caption}[numbered]
\setbeamertemplate{navigation symbols}{}

\renewcommand{\footnotesize}{\fontsize{6.10}{12}\selectfont}

\def\argmax{\operatornamewithlimits{arg\,max}}
\def\argmin{\operatornamewithlimits{arg\,min}}

%%\bibliographystyle{plain}


\title{Seminar III: R/Bioconductor}
\author{Leonardo Collado Torres \\ lcollado@lcg.unam.mx \\  Bachelor in Genomic Sciences \\ \myurl{www.lcg.unam.mx/\string~lcollado/}}
\date{
August - December, 2009
}








\usepackage{Sweave}
\begin{document}

\begin{frame}[allowframebreaks]
  \titlepage
\end{frame}

\section*{Class outline}

\begin{frame}[allowframebreaks]
  \frametitle{Working with HTS data: a \emph{simulated} case study}
  \tableofcontents[hideallsubsections]
\end{frame}

%%%%%%%%%%%%%%%%%%%%%%%%%%%%%%%%%%%%%%%%%%%%%%%%%%%%%%%%%%%%%%%%%%%%%%%%%%%
\section{Intro}

\begin{frame}[allowframebreaks]
  \frametitle{:O}
  Prepare yourself!
\end{frame}

\begin{frame}[allowframebreaks]
  \frametitle{About}
  \begin{itemize}
  \item The idea is to learn how to use R as a scripting language to call external programs such as BLAST, Velvet and Bowtie.
  \item We'll run these programs with as many default options as we can :)
  \end{itemize}
\end{frame}

\begin{frame}[allowframebreaks, fragile]
  \frametitle{For today you'll need}
  \begin{itemize}
  \item \BIOCfunction{formatdb}
  \item \BIOCfunction{blastall}\footnote{With Ubuntu use: \pl{sudo apt-get install blast2} and voilá :)}
  \item \BIOCfunction{Velvet} \\ \url{http://www.ebi.ac.uk/~zerbino/velvet/}
  \item \BIOCfunction{Bowtie} \\ \url{http://bowtie-bio.sourceforge.net/index.shtml}
  \item Of course, some Bioconductor:
\begin{Schunk}
\begin{Sinput}
> source("http://bioconductor.org/biocLite.R")
> biocLite(c("chipseq"))
\end{Sinput}
\end{Schunk}
  \item And a LINUX or UNIX environment :)
  \end{itemize}
\end{frame}

\begin{frame}[allowframebreaks]
  \frametitle{External programs}
  \begin{itemize}
  \item As you know, \BIOCfunction{BLAST} is \alert{very} used and useful to find local alignments.
  \item \BIOCfunction{Velvet} is a great program to assemble short reads into contigs.
  \item \BIOCfunction{Bowtie} is great to align short reads to a reference genome.
  \end{itemize}
\end{frame}

\section{R for scripts}
\begin{frame}[allowframebreaks, fragile]
  \frametitle{Running R scripts}
  \begin{itemize}
  \item Thanks to functions like \BIOCfunction{system}, you can use R as your scripting language.
  \item Of course, a lot of people prefer to use \pl{shell} directly.
  \item Using R can be useful to make some plots on the fly and \BIOCfunction{proc.time} helps us track the time spent running our script.
  \item You can either use: \\ \pl{R CMD BATCH file.R} or \\ \pl{Rscript file.R > file.log} as a shortcut
  \end{itemize}
\end{frame}

\begin{frame}[allowframebreaks, fragile]
  \frametitle{Use \pl{paste}}
  \begin{itemize}
  \item To build system calls, the \BIOCfunction{paste} function with the \pl{sep} or \pl{collapse} arguments is quite useful:
\begin{Schunk}
\begin{Sinput}
> args <- c(1, 2)
> call <- paste("-arg1", args[1], 
+     "-arg2", args[2], sep = " ")
> print(call)
\end{Sinput}
\begin{Soutput}
[1] "-arg1 1 -arg2 2"
\end{Soutput}
\begin{Sinput}
> call2 <- paste(c("-args", args), 
+     collapse = " ")
> print(call2)
\end{Sinput}
\begin{Soutput}
[1] "-args 1 2"
\end{Soutput}
\end{Schunk}
  \item Using a \BIOCfunction{print} coupled to a \BIOCfunction{system.time} can be useful for \alert{slow} commands.
  \end{itemize}
\end{frame}


\section{BLAST}
\begin{frame}[allowframebreaks]
  \frametitle{Command line}
  \begin{itemize}
  \item You all know how it works, and have run it through the web interface: \\ \url{http://www.ncbi.nlm.nih.gov/BLAST/}
  \item To run BLAST in command-line you mainly need two programs:
  \begin{enumerate}
  \item \alert{formatdb}: builds the database (targets)
  \item \alert{blastall}: actually runs BLAST
  \end{enumerate}
  \end{itemize}
\end{frame}

\begin{frame}[allowframebreaks]
  \frametitle{formatdb}
  Main arguments:
  \begin{itemize}
  \item \alert{-i}: the input file
  \item \alert{-p}: the type of database. Use \alert{T} for proteins or \alert{F} for nucleotides.
  \item \alert{-n}: the output name, meaning the name of the database.
  \end{itemize}
  Optional ones I use:
  \begin{itemize}
  \item \alert{-t}: the title
  \item \alert{-l}: the log file name
  \item \alert{-V}: to check the names of the targets use \alert{V}
  \end{itemize}
  For more info check:
  \begin{itemize}
  \item \pl{formatdb - -help} on the terminal
  \item \url{http://www.molbiol.ox.ac.uk/analysis_tools/BLAST/formatdb.shtml}
  \end{itemize}
\end{frame}

\begin{frame}[allowframebreaks]
  \frametitle{blastall}
  Main arguments:
  \begin{itemize}
  \item \alert{-p}: the type of BLAST to be run. BLASTP, BLASTN, \ldots
  \item \alert{-d}: the database name\footnote{For custom dbs, use the path to the db.}
  \item \alert{-i}: the input file name.
  \item \alert{-o}: the output file name.
  \end{itemize}
  Optional ones I use:
  \begin{itemize}
  \item \alert{-e}: the maximum e value allowed for the output file.
  \item \alert{-m}: the format of the output file. I like format 8 :) \myurlshort{www.molbiol.ox.ac.uk/analysis_tools/BLAST/BLAST_blastall/blastall_examples.shtml}{Click here for examples.}
  \end{itemize}
  For more info check:
  \begin{itemize}
  \item \pl{blastall - -help} on the terminal
  \item \url{http://www.molbiol.ox.ac.uk/analysis_tools/BLAST/BLAST_blastall.shtml}
  \end{itemize}
\end{frame}

\section{Velvet}
\begin{frame}[allowframebreaks]
  \frametitle{Quick intro}
  \begin{itemize}
  \item Published in 2008, \BIOCfunction{Velvet} is the most popular \emph{de novo} genome assembler for short reads such as those generated by Illumina.
  \item Its based on \emph{de Brujin graphs} and its most important parameter is the k-mer length; similar to the word size.
  \item For more info check the paper: \url{http://www.ncbi.nlm.nih.gov/pubmed/18349386}
  \end{itemize}
\end{frame}

\begin{frame}[allowframebreaks]
  \frametitle{velveth}
  In order to use Velvet we first need to run \BIOCfunction{velveth} and specify the:
  \begin{itemize}
  \item output dir: first value (without any flag)
  \item k-mer length: an integer up to 31.\footnote{The lower the value, the slower it runs.}
  \item input file format: main options are fasta and fastq.
  \item type of data: mainly either short or long.
  \item input file name
  \end{itemize}
  For more info type \pl{velveth} or check the Velvet manual.
\end{frame}

\begin{frame}[allowframebreaks]
  \frametitle{velvetg}
  After running \pl{velveth} we can run \BIOCfunction{velvetg} one or more times on the same directory. \\ Velvetg actually runs Velvet and creates the contigs. \\ To run it type \pl{velvetg} specifying:
  \begin{itemize}
  \item the output dir: again, the first unflagged value.
  \item some filtering or output options such as \pl{min\_contig\_lgth}
  \end{itemize}
  For more info type \pl{velvetg} or check the Velvet manual.
\end{frame}

\section{Bowtie}
\begin{frame}[allowframebreaks]
  \frametitle{Quick intro}
  \begin{itemize}
  \item \BIOCfunction{Bowtie} is a second generation\footnote{If you consider MAQ to be the first generation.} short read aligner that is \alert{VERY} fast.
  \item It's based on the Burrows-Wheeler Transform (BWT) as other fast aligners. Therefore, it builds an index\footnote{Similar to the BLAST database.} of the reference genome, which speeds up the process.
  \item It's very well maintained and for more info check the homepage and related paper :)
  \end{itemize}
\end{frame}

\begin{frame}[allowframebreaks]
  \frametitle{bowtie-build}
  \begin{itemize}
  \item It's very simple to use :)
  \item Just specify the input file\footnote{In FASTA format.} and the output name for the index.
  \item After building the index, move the output files\footnote{Yup, a few are created.} into \pl{PathToBowtie/indexes/}
  \end{itemize}
  For more info type: \pl{bowtie-build -h}
\end{frame}

\begin{frame}[allowframebreaks]
  \frametitle{bowtie}
  After building your index a quick way to check it is to type: \\ \pl{bowtie -c IndexName GCGTGAGCTATGAGAAAGCGCCACGCTTCC} \\ Then to run Bowtie I normally use the following arguments:
  \begin{itemize}
  \item \alert{-f}: the input file name
  \item \alert{- -all}: to force Bowtie to find all the alignments.\footnote{Obviously increases the time quite a bit on real cases.}
  \item \alert{- -al}: the output name for the FASTA file with the reads that were aligned.
  \item \alert{- -un}: the reads that did not align.
  \item Other useful arguments are \alert{-m} and \alert{- -max}.
  \end{itemize}
  For more info type \pl{bowtie -h} or check the \myurlshort{bowtie-bio.sourceforge.net/manual.shtml}{manual}.
\end{frame}

\section{Work}

\begin{frame}[allowframebreaks]
  \frametitle{Data and problem to solve}
  \begin{itemize}
  \item I generated 18 sets of 70 thousand 50bp reads. One set per student ;)\footnote{To find out which one is yours, use the order from \emph{Usuarios} at \emph{Cursos}. For example, Fonseca is number 4 and Zepeda Martinez is 17.}
  \item Imagine that these sequences come from a genome related to our species of interest. We want to find variation signatures such as: deletions, invertions and duplications.
  \item Always be open to \emph{fishy} stuff!
  \end{itemize}
\end{frame}

\begin{frame}[allowframebreaks]
  \frametitle{Part I}
  \begin{itemize}
  \item We don't know the \alert{name} of our species of interest!!!
  \item Find it out by building contigs and aligning them versus all known genomes (nucleotides).
  \item Explore\footnote{Check the files, check the alphabet by cycle frequency, \ldots} the reads that were not used to build the contigs.
  \item Conclude, remark, etc.
  \end{itemize}
\end{frame}

\begin{frame}[allowframebreaks]
  \frametitle{Part II}
  \begin{itemize}
  \item How many \alert{protein coding genes} did we cover at 90\% or greater identity and 90\% or greater query coverage?
  \item You will need to download the FASTA file with the sequence from those genes. Easy to do with the GenBank identifier :)
  \item Conclude, remark, etc.
  \end{itemize}
\end{frame}

\begin{frame}[allowframebreaks]
  \frametitle{Part III}
  \begin{itemize}
  \item Align the reads versus our the reference genome of our species of interest.
  \item Explore and compare the reads that align more than once and those that aling only once.
  \item Identify the number of deletions, duplications and inversions. Plots like \BIOCfunction{coverageplot}, \BIOCfunction{densityplot} and \BIOCfunction{stripplot} will be most useful. To use  them re-check the \BIOCfunction{chipseq} workflow :) Make some example plots and for the latter two try to make plots spanning all the genome\footnote{Only where you have reads mapped to it.}.
  \item Conclude, remark, etc.
  \end{itemize}
\end{frame}

\begin{frame}[allowframebreaks]
  \frametitle{Optional parts}
  \begin{itemize}
  \item Using the \BIOCfunction{chipseq} worflow, explore only those reads that map to more than one spot.
  \item Plot the reads using \BIOCfunction{GenomeGraphs} and add boxes for every known gene.
  \item Try to pinpoint the exact deleted, duplicated and/or inverted bases. Specially the breakpoints.
  \end{itemize}
\end{frame}

\begin{frame}[allowframebreaks]
  \frametitle{Time to work!}
  \begin{itemize}
  \item Once you are done, let me know and I'll upload all files related to your case :)
  \item Compare your conclusions with files such as \pl{segments.txt} and explore the \pl{fig} folder.
  \item The \pl{ref.fa} file is the actual \emph{reference} genome from where I got the 70k reads. Feel free to map your reads to it; some will cannot be uniquely aligned!
  \item Once everyone is done, I'll upload the \BIOCfunction{fastagen.R} script that created all the data.
  \end{itemize}
\end{frame}

\begin{frame}[allowframebreaks, fragile]
  \frametitle{SessionInfo} \scriptsize
\begin{Schunk}
\begin{Sinput}
> sessionInfo()
\end{Sinput}
\begin{Soutput}
R version 2.10.0 (2009-10-26) 
i686-pc-linux-gnu 

locale:
 [1] LC_CTYPE=en_US.UTF-8      
 [2] LC_NUMERIC=C              
 [3] LC_TIME=en_US.UTF-8       
 [4] LC_COLLATE=en_US.UTF-8    
 [5] LC_MONETARY=C             
 [6] LC_MESSAGES=en_US.UTF-8   
 [7] LC_PAPER=en_US.UTF-8      
 [8] LC_NAME=C                 
 [9] LC_ADDRESS=C              
[10] LC_TELEPHONE=C            
[11] LC_MEASUREMENT=en_US.UTF-8
[12] LC_IDENTIFICATION=C       

attached base packages:
[1] stats     graphics  grDevices
[4] utils     datasets  methods  
[7] base     
\end{Soutput}
\end{Schunk}
\end{frame}

\end{document}

