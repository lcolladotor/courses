%\VignetteIndexEntry{Programa de Seminario III: R/Bioconductor}
%\VignetteDepends{}
%\VignetteKeywords{R, Bioconductor}
%\VignettePackage{SIII: R/Bioc}
\documentclass[letterpaper,12pt]{article}

%%%%%%%%%%%%%%%%%%%%%%%% Standard Packages %%%%%%%%%%%%%%%%%%%%%%%%%%%%%%%%%%%%%%%%%%
%\usepackage{epsfig}
%\usepackage{graphicx}
%\usepackage{graphics}
%\usepackage{amssymb}
%\usepackage{amsmath}
%\usepackage{mathrsfs}
%\usepackage{caption}
%\usepackage{comment}
\usepackage{fancyvrb}
\usepackage{fancyhdr}

\usepackage[a4paper]{geometry}
\usepackage{hyperref,graphicx}

\usepackage[spanish]{babel}
\selectlanguage{spanish}
\usepackage[utf8]{inputenc}

%%%%%%%%%%%%%%%%%%%%%% headers and footers %%%%%%%%%%%%%%%%%%%%%%%%%%%%%%%%%%%%%%%%%%%%
\pagestyle{fancy} 
\renewcommand{\footrulewidth}{\headrulewidth}

%%%%%%%%%%%%%%%%%%%%%%%%% bibliography  %%%%%%%%%%%%%%%%%%%%%%%%%%%%%%%%%%%%%%%%%%%%%%%
\bibliographystyle{plainnat}

%%%%%%%%%%%%%%%%%%%%%%%%% sweave options  %%%%%%%%%%%%%%%%%%%%%%%%%%%%%%%%%%%%%%%%%%%%%%%




%%%%%%%%%%%%%%%%%%%%%%% opening %%%%%%%%%%%%%%%%%%%%%%%%%%%%%%%%%%%%
\title{\textbf{Seminario III: \texttt{R}/\texttt{Bioconductor}\\Agosto-Diciembre 2009}}
\author{Leonardo Collado Torres\\[1em]Licenciado en Ciencias Genómicas,\\ UNAM, Cuernavaca, México\\[1em]\texttt{lcollado@lcg.unam.mx}}

\usepackage{Sweave}
\begin{document}
\maketitle

\medskip
\noindent{\small\textbf{Ayudantes:} Alejandro Reyes \texttt{areyes@lcg.unam.mx}, José Reyes \texttt{jreyes@lcg.unam.mx} y Víctor Moreno \texttt{jmoreno@lcg.unam.mx}}

\medskip
\begin{abstract}
El curso Seminario III: R/Bioconductor será impartido en la Licenciatura de Ciencias Genómicas de la UNAM los viernes de 12:00 a 14:00. Dicho curso profundizará en Bioconductor que es el conjunto de herramientas públicas, montadas en \texttt{R} y desarrolladas para el estudio de la genómica. Para tomar esta clase es necesario tener un manejo básico de estadística por un lado y de R por el otro. En el segundo caso se impartió un curso de \texttt{R} introductorio a la sexta generación de la LCG disponible en \url{http://www.lcg.unam.mx/~lcollado/E/}. El orden en el que se cubrirá el material y el proyecto asociado a esta materia estarán intregrados y directamente relacionados al curso de Bioinformática y Estadística I. 
\end{abstract}

\section{Objetivos}

\section{Proyecto}

\section{Evaluación}
% Mover despues de programa de clases?

\section{Programa inicial de clases}

\begin{itemize} 

\item[14 Ago] Clase I
  \begin{description}
  \item[Introducción y R básico] conociendo el lenguaje
  \begin{enumerate}
  \item Orígenes de R
  \end{enumerate}
  \end{description}
  
  \item[21 Ago] Clase II
  \begin{description}
  \item[Intro a R] parte II
  \begin{enumerate}
  \item Estructuras de Control
  \end{enumerate}
  \end{description}
  
  \item[28 Ago] Clase III
  \begin{description}
  \item[Fin de la intro a R] parte III
  \begin{enumerate}
  \item Dudas
  \end{enumerate}
  \end{description}
  
  \item[4 Sept] Clase IV
  \begin{description}
  \item[Estadística Descriptiva] Empieza tema de variables aleatorias discretas
  \begin{enumerate}
  \item Tabla de Contingencia (table)
  \end{enumerate}
  \end{description}
  
  \item[11 Sept] Clase V
  \begin{description}
  \item[Estadística Descriptiva] Visualización general de datos
  \begin{enumerate}
  \item Histogramas de frecuencias y distribución de frecuencias acumuladas (hist)
  \end{enumerate}
  \end{description}
  
  \item[23 Feb] Clase VI
  \begin{description}
  \item[Más de gráficas] Mosaicos
  \begin{enumerate}
  \item Detalles gráficos o distribuciones (rbinom, rpois, rgeom)
  \end{enumerate}
  \end{description}
  
  \item[18 Sept] Clase VII
  \begin{description}
  \item[Primer trabajo aplicado] Por definir
  \begin{enumerate}
  \item Describir trabajo aplicado?
  \end{enumerate}
  \end{description}
  
  \item[25 Sept] Clase VIII
  \begin{description}
  \item[Título Clase] Variables aleatorias continuas
  \begin{enumerate}
  \item por ver
  \end{enumerate}
  \end{description}
  
  \item[2 Oct] Clase IX
  \begin{description}
  \item[Título Clase] VAC II
  \begin{enumerate}
  \item por ver
  \end{enumerate}
  \end{description}
  
  \item[9 Oct] Clase X
  \begin{description}
  \item[Título Clase] VAC III
    \begin{enumerate}
      \item por ver
    \end{enumerate}
  \end{description}
  
  \item[16 Oct] Clase XI
  \begin{description}
  \item[Título Clase] VAC IV
  \begin{enumerate}
  \item por ver
  \end{enumerate}
  \end{description}
  
  \item[23 Oct] Clase XII
  \begin{description}
  \item[Título Clase] VAC V
  \begin{enumerate}
  \item por ver
  \end{enumerate}
  \end{description}  
  
  \item[30 Oct] Clase XIII
  \begin{description}
  \item[Título Clase] VAC VI
  \begin{enumerate}
  \item por ver
  \end{enumerate}
  \end{description}
  
  \item[6 Nov] Clase XIV
  \begin{description}
  \item[Título Clase] VAC VII
  \begin{enumerate}
  \item por ver
  \end{enumerate}
  \end{description}  
    
  \item[13 Nov] Clase XV
  \begin{description}
  \item[Título Clase] Distribución de probabilidad multivariable
  \begin{enumerate}
  \item por ver
  \end{enumerate}
  \end{description}
  
  \item[20 Nov] Clase XVI
  \begin{description}
  \item[Título Clase] DPM II
  \begin{enumerate}
  \item por ver
  \end{enumerate}
  \end{description}
    
  \item[27 Nov] Clase XVII
  \begin{description}
  \item[Título Clase] DPM III
  \begin{enumerate}
  \item por ver
  \end{enumerate}
  \end{description}
  
  \item[4 Dic] Primera Semana de Exámenes
  \begin{description}
  \item[Título Clase] DPM IV
  \begin{enumerate}
  \item por ver
  \end{enumerate}
  \end{description}
  
  \item[11 Dic] Segunda Semana de Exámenes
  \begin{description}
  \item[Título Clase] Intervalos de Convianza
  \begin{enumerate}
  \item por ver
  \end{enumerate}
  \end{description}

  
\end{itemize}

\end{document}
