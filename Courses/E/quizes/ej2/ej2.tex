\documentclass[letterpaper,12pt]{article}

%%%%%%%%%%%%%%%%%%%%%%%% Standard Packages %%%%%%%%%%%%%%%%%%%%%%%%%%%%%%%%%%%%%%%%%%
\usepackage{epsfig}
\usepackage{graphicx}
\usepackage{graphics}
\usepackage{amssymb}
\usepackage{amsmath}
\usepackage{mathrsfs}
\usepackage{fancyvrb}
\usepackage{caption}
\usepackage{comment}
\usepackage{fancyhdr}
%Use the latin1 encoding http://fontignie.blogspot.com/2006/04/accents-in-latex.html
\usepackage[latin1]{inputenc}
%accents
\usepackage[cyr]{aeguill}
%yes, I am french
\usepackage[spanish]{babel}

%%%%%%%%%%%%%%%%%%%%%%%% Adapted from Sweave %%%%%%%%%%%%%%%%%%%%%%%%%%%%%%%%%%%%%%%%%

\DefineVerbatimEnvironment{Rcode}{Verbatim}{fontshape=sl, frame=single, 
  framesep=2mm, fontsize=\small, baselinestretch=.5}

%%%%%%%%%%%%%%%%%%%%%%%%%%%%%%%%%%%%%%%%%%%%%%%%%%%%%%%%%%%%%%%%%%%%%%%%%%%%%%%%%%%%%%

%%%%%%%%%%%%%%%%%%%%%%%% My macros (which of course are borrowed from a million ... %%
\def\argmax{\operatornamewithlimits{arg\,max}}
\def\argmin{\operatornamewithlimits{arg\,min}}


%%%%%%%%%%%%%%%%%%%%%%%% Page and Document Setup %%%%%%%%%%%%%%%%%%%%%%%%%%%%%%%%%%%%%
\addtolength{\oddsidemargin}{-0.875in}
\addtolength{\topmargin}{-0.875in}
\addtolength{\textwidth}{1.75in}
\addtolength{\textheight}{1.75in}

\captionsetup{margin=15pt,font=small,labelfont=bf}

\renewcommand{\topfraction}{0.9}        % max fraction of floats at top
\renewcommand{\bottomfraction}{0.8}     % max fraction of floats at bottom

% Parameters for TEXT pages (not float pages):
\setcounter{topnumber}{2}
\setcounter{bottomnumber}{2}
\setcounter{totalnumber}{4}             % 2 may work better
\setcounter{dbltopnumber}{2}            % for 2-column pages
\renewcommand{\dbltopfraction}{0.9}     % fit big float above 2-col. text
\renewcommand{\textfraction}{0.07}      % allow minimal text w. figs

% Parameters for FLOAT pages (not text pages):
\renewcommand{\floatpagefraction}{0.7}          % require fuller float pages

% N.B.: floatpagefraction MUST be less than topfraction !!
\renewcommand{\dblfloatpagefraction}{0.7}       % require fuller float pages


%%%%%%%%%%%%%%%%%%%%%%% options for sweave %%%%%%%%%%%%%%%%%%%%%%%%%%%%%%%%%%%%%%%%%%%%


%%%%%%%%%%%%%%%%%%%%%% headers and footers %%%%%%%%%%%%%%%%%%%%%%%%%%%%%%%%%%%%%%%%%%%%
\pagestyle{fancy} 
\renewcommand{\footrulewidth}{\headrulewidth}

%%%%%%%%%%%%%%%%%%%%%%%%% bibliography  %%%%%%%%%%%%%%%%%%%%%%%%%%%%%%%%%%%%%%%%%%%%%%%
\bibliographystyle{plainnat}


%%%%%%%%%%%%%%%%%%%%%%%%%%%%%%%%%%%%%%%%%%%%%%%%%%%%%%%%%%%%%%%%%%%%%%%%%%%%%%%%%%%%%%%
%%%%%%%%%%%%%%%%%%%%%%%%% Now Edit %%%%%%%%%%%%%%%%%%%%%%%%%%%%%%%%%%%%%%%%%%%%%%%%%%%%
%%%%%%%%%%%%%%%%%%%%%%%%%%%%%%%%%%%%%%%%%%%%%%%%%%%%%%%%%%%%%%%%%%%%%%%%%%%%%%%%%%%%%%%
\fancyhead[L]{\em Leonardo Collado Torres y Mar�a Guti�rrez Arcelus}
\fancyhead[R]{\em Ejercicios de Introducci�n a R: parte II}
\fancyfoot[L]{\em \today}
\fancyfoot[C]{}
\fancyfoot[R]{\em \thepage}


%%%%%%%%%%%%%%%%%%%%%%% opening %%%%%%%%%%%%%%%%%%%%%%%%%%%%%%%%%%%%
\title{Ejercicios de Introducci�n a R: parte II}
\author{Leonardo Collado Torres y Mar�a Guti�rrez Arcelus}

\usepackage{Sweave}
\begin{document}
\maketitle
\begin{enumerate} 
  \item Usen el conjunto de datos \texttt{islands} que viene en \texttt{R}. Este contiene el tama�o de los bloques de tierra cuya �rea excede las 10 mil millas cuadradas. Encuentren las 7 m�s grandes y las 2 m�s chicas.
  Les recomendamos que usen lo visto en �ndices de vectores y la funci�n \texttt{sort}.
  \bigskip
  
  \item Con \texttt{read.table} lean la tabla de datos '10biggestPhages.txt' \footnote{Chequen la presentaci�n si tienen dudas de como leer la tabla.} y encuentren lo siguiente:
  \begin{enumerate}
	\item \textquestiondown{}Todos los "Taxid" son �nicos? Usen la funci�n \texttt{unique}.
	\item \textquestiondown{}C�al es el tama�o de genoma m�s chico en la tabla? Usen la funci�n \texttt{sort} o la funci�n \texttt{min}.
	\item \textquestiondown{}C�al es el nombre de ese fago? Les puede ayudar la funci�n \texttt{which}.
  \end{enumerate}
  \bigskip
  
  \item \textquestiondown{}C�antas combinaciones por pares diferentes pueden hacer con Fernando, Sur y Mariana? :P

En \texttt{R} podemos usar la funci�n \texttt{combn} para hacer combinaciones.
\begin{Schunk}
\begin{Sinput}
> combn(c("Fer", "Sur", "Mariana"), 2)
\end{Sinput}
\end{Schunk}
  \bigskip
  
  \item Digamos que al ir a una fiesta puedes salir en terminar en 4 estados diferentes:
  \begin{itemize}
    \item 'Sobrio' con probabilidad de 0.2
	\item 'Happy' con probabilidad de 0.4
	\item 'Pedo' con probabilidad de 0.3
	\item 'Muerto' con probabilidad de 0.1
  \end{itemize}
Respondan las siguientes preguntas:
  \begin{enumerate}
    \item \textquestiondown{}C�al es la probabilidad de que en la fiesta 1 alguien termine 'Happy' y en la fiesta 2 esa misma persona termine 'Pedo'?  
	\item \textquestiondown{}C�al es la probabilidad de que en la fiesta 2 alguien termine 'Muerto' dado que en la fiesta 1 estuvo 'Sobrio'?
  \end{enumerate}
Pueden apoyarse en la siguiente tabla que generamos con R para responder las preguntas.
\begin{Schunk}
\begin{Sinput}
> fiesta.1 <- c(0.2, 0.4, 0.3, 0.1)
> names(fiesta.1) <- c("Sobrio", "Happy", "Pedo", "Muerto")
> fiesta.2 <- NULL
> for (i in 1:4) fiesta.2 <- rbind(fiesta.2, fiesta.1 * fiesta.1[i])
> rownames(fiesta.2) <- names(fiesta.1)
> fiesta.2
\end{Sinput}
\begin{Soutput}
       Sobrio Happy Pedo Muerto
Sobrio   0.04  0.08 0.06   0.02
Happy    0.08  0.16 0.12   0.04
Pedo     0.06  0.12 0.09   0.03
Muerto   0.02  0.04 0.03   0.01
\end{Soutput}
\end{Schunk}
  \bigskip
  
\end{enumerate}


\end{document}
