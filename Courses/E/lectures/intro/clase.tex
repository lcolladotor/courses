%%%%%%%%%%%%%%%%%%%%%%%%%%%%%%%%%%%%%%%%%%%%%%%%%%%%%%%%%%%%%%%%%%%%%%%%%%%
%%%%%%%%%%%%%%%%%%%%%%%%%%%%%%%%%%%%%%%%%%%%%%%%%%%%%%%%%%%%%%%%%%%%%%%%%%%
%%
%% The basic tex file for the header of all the lectures. 
%%

\documentclass{beamer}
\usetheme[hideallsubsections]{Berkeley}

\usepackage{color}
\usepackage{amsfonts}

\definecolor{myblue}{rgb}{0.25, 0, 0.75}
\definecolor{mygold}{rgb}{1,0.8,0.2}
\definecolor{gray}{rgb}{0.5, 0.5, 0.5}
\definecolor{lucia}{rgb}{0.8,0.4,0.7} 

\newcommand{\myurl}[1]{\href{http://#1}{\textcolor{gray}{\texttt{#1}}}}
\newcommand{\myem}[1]{\structure{#1}}
\newcommand{\RPack}[1]{\textcolor{gray}{\textsf{#1}}}
\newcommand{\pl}[1]{\texttt{#1}}
\newcommand{\Rcode}[1]{\texttt{#1}}
\newcommand{\Rfunction}[1]{\href{http://www.statmethods.net/search/index.asp?QU=#1&search=Search&Action=Search}{\textcolor{orange}{\textsf{#1}}}}
\newcommand{\myurlshort}[2]{\href{http://#1}{\textcolor{gray}{\textsf{#2}}}}
\newcommand{\RClass}[1]{\textcolor{mygold}{\textsf{#1}}}
\newcommand{\BIOCfunction}[1]{\textcolor{orange}{#1}}

\setbeamercolor{example text}{fg=lucia}
\setbeamertemplate{sections/subsections in toc}[ball unumbered]
\setbeamertemplate{frametitle continuation}[from second][]
\setbeamertemplate{itemize subitem}[triangle]
\setbeamertemplate{footline}[page number]
\setbeamertemplate{caption}[numbered]

\renewcommand{\footnotesize}{\fontsize{6.10}{12}\selectfont}

\def\argmax{\operatornamewithlimits{arg\,max}}
\def\argmin{\operatornamewithlimits{arg\,min}}


%%%%%%%%%%%%%%%%%%%%%%%%%%%%%%%%%%%%%%%%%%%%%%%%%%%%%%%%%%%%%%%%%%%%%%%%%%%
\title{R / Bioconductor: Curso Intensivo}

\author[]{\myem{Leonardo Collado Torres}\\
  Licenciatura en Ciencias Gen�micas, UNAM\\
  \myurl{www.lcg.unam.mx/\string~lcollado/index.php}\\
}

\date{
  Cuernavaca, M�xico\\
  Oct-Nov, 2008
}






\usepackage{Sweave}
\begin{document}

%%% set up some options for Sweave and R %%%

%%%%%%%%%%%%%%%%%%%%%%%%%%%%%%%%%%%%%%%%%%%%%%%%%%%%%%%%%%%%%%%%%%%%%%%%%%%
%%%%%%%%%%%%%%%%%%%%%%%%%%%%%%%%%%%%%%%%%%%%%%%%%%%%%%%%%%%%%%%%%%%%%%%%%%%
\begin{frame}[allowframebreaks]
  \titlepage
\end{frame}

\begin{frame}[allowframebreaks]
  \frametitle{Introducci�n y R b�sico}
  \tableofcontents
\end{frame}

%%%%%%%%%%%%%%%%%%%%%%%%%%%%%%%%%%%%%%%%%%%%%%%%%%%%%%%%%%%%%%%%%%%%%%%%%%%
\section{Or�genes}

\begin{frame}[allowframebreaks]
  \frametitle{De donde viene R}
  \begin{itemize}
  \item Para muchos \pl{R} es un dialecto porque es un derivado del lenguaje \pl{S} creado por John Chambers y co en los Bell Labs. En si, \pl{R} fue escrito a mitad de los 90s por Ross Ihaka y Robert Gentleman.
  \item Desde 1997, \pl{R} ha sido manejado por el \emph{R Development Core Team} y se ha mantenido como open-source.
  \item Una ventaja de R es que se puede usar en varias plataformas: UNIX, Windows, Mac. 
  \item \pl{R} en si es un lenguaje de computaci�n creado para facilitar la manipulaci�n de datos, hacer c�lculos y gr�ficas de alto nivel. Es por esto que \pl{R} es fuerte en estad�stica.
  \end{itemize}
\end{frame}

\begin{frame}[allowframebreaks]
  \frametitle{Propiedades de R}
  \pl{R} es un ambiente para trabajar en estad�stica computacional y al mismo tiempo es un lenguaje de programaci�n. Hay usuarios que solo van a usar las funciones b�sicas de \pl{R} (como una calculadora) mientras otros incluso har�n paquetes que ligen \pl{R} con \pl{C}. En fin, \pl{R}:
  \begin{itemize}
  \item es efectivo en el manejo de datos y su almacenamiento.
  \item tiene muchos operadores para hacer c�lculos en arreglos (vectores) y matrices.
  \item tiene una gama de herramientas para el an�lisis de datos. Hay muchos paquetes disponibles, como la familia de Bioconductor.
  \item tiene un sistema de gr�ficas muy �til para el an�lisis de datos. Excel es cosa del pasado ;)
  \item ya viene con modelos estad�sticos.
  \item hay muchos manuales y un sistema de ayuda bastante bueno. Adem�s hay una comunidad internacional que te puede ayudar :).
  \end{itemize}
\end{frame}

%%%%%%%%%%%%%%%%%%%%%%%%%%%%%%%%%%%%%%%%%%%%%%%%%%%%%%%%%%%%%%%%%%%%%%%%%%%
\section{Uso b�sico de R}

\begin{frame}[allowframebreaks]
  \frametitle{Abrir y cerrar R}
  \begin{itemize}
  \item Para abrir \pl{R} simplemente tienen que escribir el comando \Rcode{R} en UNIX. Lo primero que veran es una peque�a descripci�n de \pl{R} incluyendo la versi�n que tienen instalada.
  \item Al abrir \pl{R}, este busca en el directorio donde est�n informaci�n de alguna sesi�n previa. Esto luego sera �til con los \emph{workspace}.
  \item Para cerrar \pl{R} simplemente escriban \BIOCfunction{q()}. Les va a pedir si quieren guardar una imagen del workspace -- por ahora digan que no. 
  \end{itemize}
\end{frame}

\begin{frame}[allowframebreaks]
  \frametitle{Workspace}
  \begin{itemize}
  \item Muchas veces tienes que interrumpir tu trabajo. \pl{R} tiene toda una funcionalidad llamada workspace que te ayuda a retomar tu trabajo de sesiones previas.
  \item Cuando guardas el workspace se crean dos archivos: \pl{.RData} y \pl{.Rhistory} en el directorio donde estes trabajando. Estos almacenan todos los objetos que haya definido el usuario (vectores, matrices, listas, funciones). La pr�xima vez que abras R en ese directorio, carga todo lo que creaste antes autom�ticamente.
  \item Hay una serie de funciones que les pueden ayudar para organizar su trabajo en R. \Rfunction{getwd} te da tu directorio de trabajo actual, \Rfunction{setwd} lo cambia, \Rfunction{history} te muestra los �ltimos 25 comandos que usaste y \Rcode{history(max.show=Inf)} te muestra todos.
  \item Otras funciones �tiles son \Rfunction{savehistory} y \Rfunction{loadhistory}. Adem�s si quieres ver que objetos tienes en tu sesi�n puedes usar \Rfunction{objects} o \Rfunction{ls}; puedes quitar objectos con \Rfunction{rm}.
  \item Si quieren guardar el workspace manualmente o con un nombre diferente a \pl{.Rdata} usen \BIOCfunction{\Rcode{save.image()}}. De igual forma, pueden cargar un workspace manualmente con \Rfunction{load}.
  \item Para checar las opciones del ambiente de \pl{R} usen \Rfunction{options}. Por ejemplo, pueden cambiar cuantos d�gitos se imprimen en el output.
  \end{itemize}
\end{frame}  

\begin{frame}[fragile, allowframebreaks]
  \frametitle{R como Calculadora}
  \begin{itemize}
  \item R es un \emph{expression language}. Aka\footnote{also known as}, una R no es igual a una r. Los nombres de variables tienen que empezar por un punto\footnote{Una letra le tiene que seguir al punto para que sea un nombre v�lido} o caracteres alfanum�ricos.
  \end{itemize}
\begin{Schunk}
\begin{Sinput}
> 2 + 2
> 2^2
> r <- c(1:3, 4.5, 109)
> pi * r^2
> sqrt(36)
> sin(2 * pi)
> exp(1)
> log(10)
> log(10, base = 10)
\end{Sinput}
\end{Schunk}
\end{frame}

\begin{frame}[fragile, allowframebreaks]
  \frametitle{Asignaci�n de valores}
  \begin{itemize}
  \item En R, hay 3 formas de asignar valores, aunque en general se usan solo dos: \Rcode{=} y \Rcode{<-}
  \item Preferencialmente usen \BIOCfunction{\Rcode{<-}} simplemente para evitar confusiones. Es que el signo \Rcode{=} se usa para el paso de valores en las funciones.
  \end{itemize}
\begin{Schunk}
\begin{Sinput}
> A <- c(a = 1, b = 2)["b"]
> A = c(a = 1, b = 2)["b"]
> A
\end{Sinput}
\begin{Soutput}
b 
2 
\end{Soutput}
\end{Schunk}
  \begin{itemize}
  \item Aqu� queda m�s clara la asignaci�n en la primera l�nea, aunque las dos hacen lo mismo.
  \end{itemize}
\end{frame}

%%%%%%%%%%%%%%%%%%%%%%%%%%%%%%%%%%%%%%%%%%%%%%%%%%%%%%%%%%%%%%%%%%%%%%%%%%%

\end{document}

