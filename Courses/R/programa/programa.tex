\documentclass[letterpaper,12pt]{article}

%%%%%%%%%%%%%%%%%%%%%%%% Standard Packages %%%%%%%%%%%%%%%%%%%%%%%%%%%%%%%%%%%%%%%%%%
\usepackage{epsfig}
\usepackage{graphicx}
\usepackage{graphics}
\usepackage{amssymb}
\usepackage{amsmath}
\usepackage{mathrsfs}
\usepackage{fancyvrb}
\usepackage{caption}
\usepackage{comment}
\usepackage{fancyhdr}

%%%%%%%%%%%%%%%%%%%%%%%% Adapted from Sweave %%%%%%%%%%%%%%%%%%%%%%%%%%%%%%%%%%%%%%%%%

\DefineVerbatimEnvironment{Rcode}{Verbatim}{fontshape=sl, frame=single, 
  framesep=2mm, fontsize=\small, baselinestretch=.5}

%%%%%%%%%%%%%%%%%%%%%%%%%%%%%%%%%%%%%%%%%%%%%%%%%%%%%%%%%%%%%%%%%%%%%%%%%%%%%%%%%%%%%%

%%%%%%%%%%%%%%%%%%%%%%%% My macros (which of course are borrowed from a million ... %%
\def\argmax{\operatornamewithlimits{arg\,max}}
\def\argmin{\operatornamewithlimits{arg\,min}}


%%%%%%%%%%%%%%%%%%%%%%%% Page and Document Setup %%%%%%%%%%%%%%%%%%%%%%%%%%%%%%%%%%%%%
\addtolength{\oddsidemargin}{-0.875in}
\addtolength{\topmargin}{-0.875in}
\addtolength{\textwidth}{1.75in}
\addtolength{\textheight}{1.75in}

\captionsetup{margin=15pt,font=small,labelfont=bf}

\renewcommand{\topfraction}{0.9}        % max fraction of floats at top
\renewcommand{\bottomfraction}{0.8}     % max fraction of floats at bottom

% Parameters for TEXT pages (not float pages):
\setcounter{topnumber}{2}
\setcounter{bottomnumber}{2}
\setcounter{totalnumber}{4}             % 2 may work better
\setcounter{dbltopnumber}{2}            % for 2-column pages
\renewcommand{\dbltopfraction}{0.9}     % fit big float above 2-col. text
\renewcommand{\textfraction}{0.07}      % allow minimal text w. figs

% Parameters for FLOAT pages (not text pages):
\renewcommand{\floatpagefraction}{0.7}          % require fuller float pages

% N.B.: floatpagefraction MUST be less than topfraction !!
\renewcommand{\dblfloatpagefraction}{0.7}       % require fuller float pages

%%%%%%%%%%%%%%%%%%%%%% headers and footers %%%%%%%%%%%%%%%%%%%%%%%%%%%%%%%%%%%%%%%%%%%%
\pagestyle{fancy} 
\renewcommand{\footrulewidth}{\headrulewidth}

%%%%%%%%%%%%%%%%%%%%%%%%% bibliography  %%%%%%%%%%%%%%%%%%%%%%%%%%%%%%%%%%%%%%%%%%%%%%%
\bibliographystyle{plainnat}



%%%%%%%%%%%%%%%%%%%%%%% opening %%%%%%%%%%%%%%%%%%%%%%%%%%%%%%%%%%%%
\title{Curso sobre R/Bioconductor -- Cuernavaca, Oct-Nov 2008}
\author{Leonardo Collado y Osam Yanez}

\begin{document}
\maketitle

\begin{itemize} 
\item[28-29 Oct] Primera Clase
  \begin{description}
  \item[Introduccion y R basico] conociendo el lenguaje
    \begin{enumerate}
      \item Usar R, workspace
      \item Como obtener ayuda
	  \item Estructuras de control
	  \item Leer un archivo, directorio
      \item Definir una funcion
	  \item Como instalar paquetes
    \end{enumerate}
  \end{description}

\item[30-31 Oct] Segunda Clase
  \begin{description}
  \item[Datos Univariados] R y estadistica
    \begin{enumerate}
      \item apply, sapply, ...
      \item list
      \item funciones basicas
    \end{enumerate}
  \item[Graficas] lo basico
	\begin{enumerate}
	  \item graficas de lineas
	  \item graficas de barras
	  \item histogramas
	  \item misc
	\end{enumerate}
  \end{description}

\item[4-5 Nov] Tercera clase
  \begin{description}
  \item[Cont. Estadistica] Datos bi y multivariados
    \begin{enumerate}
      \item Scatter Plots
      \item Ejercicio tipo Iris
      \item Boxplots
    \end{enumerate}
  \item[Lattice] un paquete util para graficas
    \begin{enumerate}
      \item Introduccion rapida al paquete
	  \item Ejercicio con Chem97
      \item Ejercicios basados en el lab de lattice
    \end{enumerate}
  \end{description}

\item[6-7 Nov] Cuarta clase
  \begin{description}
  \item[Pruebas de Estadistica] un poco de parametricas y no parametricas
    \begin{enumerate}
      \item F
	  \item Chi
      \item Kolmogorov-Smirnoff
      \item Wilcoxon
	  \item Otras pruebas
	  \item (tentativo) Intervalos de confianza, ANOVAs
    \end{enumerate}
  \item[limma] Regresiones lineales con este paquete
    \begin{enumerate}
    \item Una intro al paquete
    \item Ejercicios del lab de limma
    \end{enumerate}
  \end{description}

\item[11-12 Nov] Quinta clase
  \begin{description}
  \item[Bioconductor] vamos de lleno sobre BioC 
    \begin{enumerate}
      \item Panorama general de los paquetes disponibles
      \item (tentativo) Ejercicio sobre GOs
	  \item (tentativo) Paquete de mcmc
      \item Microarreglos
    \end{enumerate}
  \end{description}

\item[13-14 Nov] Sexta clase
  \begin{description}
  \item[Clustering] revisar varias formas de hacerlo 
    \begin{enumerate}
      \item Una intro basica al clustering
      \item hclust
      \item kmeans
      \item Ejercicio para compararlos
    \end{enumerate}
   \item[Mas paquetes de R/BioC] si nos queda tiempo
     \begin{enumerate}
	   \item Paquete de DBI para R
	   \item GeneR
	   \item Un paquete que une varias DBs
	 \end{enumerate}
  \end{description}

	
\end{itemize}

\end{document}