\documentclass[letterpaper,12pt]{article}

%%%%%%%%%%%%%%%%%%%%%%%% Standard Packages %%%%%%%%%%%%%%%%%%%%%%%%%%%%%%%%%%%%%%%%%%
\usepackage{epsfig}
\usepackage{graphicx}
\usepackage{graphics}
\usepackage{amssymb}
\usepackage{amsmath}
\usepackage{mathrsfs}
\usepackage{fancyvrb}
\usepackage{caption}
\usepackage{comment}
\usepackage{fancyhdr}

%%%%%%%%%%%%%%%%%%%%%%%% Adapted from Sweave %%%%%%%%%%%%%%%%%%%%%%%%%%%%%%%%%%%%%%%%%

\DefineVerbatimEnvironment{Rcode}{Verbatim}{fontshape=sl, frame=single, 
  framesep=2mm, fontsize=\small, baselinestretch=.5}

%%%%%%%%%%%%%%%%%%%%%%%%%%%%%%%%%%%%%%%%%%%%%%%%%%%%%%%%%%%%%%%%%%%%%%%%%%%%%%%%%%%%%%

%%%%%%%%%%%%%%%%%%%%%%%% My macros (which of course are borrowed from a million ... %%
\def\argmax{\operatornamewithlimits{arg\,max}}
\def\argmin{\operatornamewithlimits{arg\,min}}


%%%%%%%%%%%%%%%%%%%%%%%% Page and Document Setup %%%%%%%%%%%%%%%%%%%%%%%%%%%%%%%%%%%%%
\addtolength{\oddsidemargin}{-0.875in}
\addtolength{\topmargin}{-0.875in}
\addtolength{\textwidth}{1.75in}
\addtolength{\textheight}{1.75in}

\captionsetup{margin=15pt,font=small,labelfont=bf}

\renewcommand{\topfraction}{0.9}        % max fraction of floats at top
\renewcommand{\bottomfraction}{0.8}     % max fraction of floats at bottom

% Parameters for TEXT pages (not float pages):
\setcounter{topnumber}{2}
\setcounter{bottomnumber}{2}
\setcounter{totalnumber}{4}             % 2 may work better
\setcounter{dbltopnumber}{2}            % for 2-column pages
\renewcommand{\dbltopfraction}{0.9}     % fit big float above 2-col. text
\renewcommand{\textfraction}{0.07}      % allow minimal text w. figs

% Parameters for FLOAT pages (not text pages):
\renewcommand{\floatpagefraction}{0.7}          % require fuller float pages

% N.B.: floatpagefraction MUST be less than topfraction !!
\renewcommand{\dblfloatpagefraction}{0.7}       % require fuller float pages


%%%%%%%%%%%%%%%%%%%%%%% options for sweave %%%%%%%%%%%%%%%%%%%%%%%%%%%%%%%%%%%%%%%%%%%%


%%%%%%%%%%%%%%%%%%%%%% headers and footers %%%%%%%%%%%%%%%%%%%%%%%%%%%%%%%%%%%%%%%%%%%%
\pagestyle{fancy} 
\renewcommand{\footrulewidth}{\headrulewidth}

%%%%%%%%%%%%%%%%%%%%%%%%% bibliography  %%%%%%%%%%%%%%%%%%%%%%%%%%%%%%%%%%%%%%%%%%%%%%%
\bibliographystyle{plainnat}


%%%%%%%%%%%%%%%%%%%%%%%%%%%%%%%%%%%%%%%%%%%%%%%%%%%%%%%%%%%%%%%%%%%%%%%%%%%%%%%%%%%%%%%
%%%%%%%%%%%%%%%%%%%%%%%%% Now Edit %%%%%%%%%%%%%%%%%%%%%%%%%%%%%%%%%%%%%%%%%%%%%%%%%%%%
%%%%%%%%%%%%%%%%%%%%%%%%%%%%%%%%%%%%%%%%%%%%%%%%%%%%%%%%%%%%%%%%%%%%%%%%%%%%%%%%%%%%%%%
\fancyhead[L]{\em Leonardo Collado Torres}
\fancyhead[R]{\em ej1}
\fancyfoot[L]{\em \today}
\fancyfoot[C]{}
\fancyfoot[R]{\em \thepage}


%%%%%%%%%%%%%%%%%%%%%%% opening %%%%%%%%%%%%%%%%%%%%%%%%%%%%%%%%%%%%
\title{Ejercicios 2: ahora s� es de tarea}
\author{Leonardo Collado Torres}

\usepackage{Sweave}
\begin{document}
\maketitle
\begin{enumerate}
  \item Usen el set de datos \texttt{DDT} del paquete \texttt{MASS}. Este contiene mediciones independientes del pesticida DDT en Brassica oleracea.
  \begin{enumerate}
    \item Hagan un histograma y un boxplot de los datos con colores (a su gusto) y t�tulos apropiados.
	\item De esta gr�ficas estimen la media y desviaci�n est�ndar.
	\item Agreguen la informaci�n del punto anterior al c�digo e impr�manla usando \texttt{print}.
	\item Tienen que subir el c�digo para responder a esta pregunta. El c�digo debe ser portable!!!!
  \end{enumerate}
  \bigskip
  \bigskip
  
  \item Revisen el par�metro \texttt{mfrow} de la funci�n \texttt{par}. Usen los datos de \texttt{lawsuits} del paquete \texttt{UsingR}. 
  \begin{enumerate}
    \item Tienen que guardar las gr�ficas en un archivo PDF para este ejercicio.
	\item En el lugar de 1 sola gr�fica, quiero que me pongan lado a lado el histograma de \texttt{lawsuits} y el histograma de los valores logar�tmicos de los mismos datos.
	\item El t�tulo de las gr�ficas debe incluir su nombre en formato de su username del email: por ejemplo, lcollado.
	\item Evidentemente, sus histogramas deben estar personalizados (colores, etc).
	\item Solo suban el c�digo (portable!!) que crea el archivo PDF con las 2 gr�ficas.
  \end{enumerate}	
  \bigskip
  \bigskip
  
  \item Simplemente busquen una gr�fica\footnote{Evidentemente, busquen una del tipo que vimos en clase}\footnote{No acepto gr�ficas tipo "pie" o "sope" como unos les llaman} en alg�n periodico o revista cient�fica y hagan una r�plica (a su estilo) usando \texttt{R}.
  \begin{itemize}
    \item Incluyan el URL de donde encontraron la gr�fica en su c�digo como comentario.
	\item Envien solo el c�digo para hacer la gr�fica y no la gr�fica en s�. Este debe ser portable!!!!\footnote{Si usan alg�n paquete que no mencion� como los de base en el curso en el foro, incluyan el c�digo para bajarlo}
	\item Si usan alg�n archivo para crear su \texttt{data set} este debe estar en su folder de \texttt{public\_html} y su codigo debe accesarlo via Internet. Otra vez... su c�digo debe ser portable!!!
    \item Suban su c�digo a la p�gina de cursos (incluyendo el de las anteriores preguntas).
  \end{itemize}
  \bigskip
  \bigskip
\end{enumerate}


\end{document}
