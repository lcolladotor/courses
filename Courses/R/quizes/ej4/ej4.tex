\documentclass[letterpaper,12pt]{article}

%%%%%%%%%%%%%%%%%%%%%%%% Standard Packages %%%%%%%%%%%%%%%%%%%%%%%%%%%%%%%%%%%%%%%%%%
\usepackage{epsfig}
\usepackage{graphicx}
\usepackage{graphics}
\usepackage{amssymb}
\usepackage{amsmath}
\usepackage{mathrsfs}
\usepackage{fancyvrb}
\usepackage{caption}
\usepackage{comment}
\usepackage{fancyhdr}

%%%%%%%%%%%%%%%%%%%%%%%% Adapted from Sweave %%%%%%%%%%%%%%%%%%%%%%%%%%%%%%%%%%%%%%%%%

\DefineVerbatimEnvironment{Rcode}{Verbatim}{fontshape=sl, frame=single, 
  framesep=2mm, fontsize=\small, baselinestretch=.5}

%%%%%%%%%%%%%%%%%%%%%%%%%%%%%%%%%%%%%%%%%%%%%%%%%%%%%%%%%%%%%%%%%%%%%%%%%%%%%%%%%%%%%%

%%%%%%%%%%%%%%%%%%%%%%%% My macros (which of course are borrowed from a million ... %%
\def\argmax{\operatornamewithlimits{arg\,max}}
\def\argmin{\operatornamewithlimits{arg\,min}}


%%%%%%%%%%%%%%%%%%%%%%%% Page and Document Setup %%%%%%%%%%%%%%%%%%%%%%%%%%%%%%%%%%%%%
\addtolength{\oddsidemargin}{-0.875in}
\addtolength{\topmargin}{-0.875in}
\addtolength{\textwidth}{1.75in}
\addtolength{\textheight}{1.75in}

\captionsetup{margin=15pt,font=small,labelfont=bf}

\renewcommand{\topfraction}{0.9}        % max fraction of floats at top
\renewcommand{\bottomfraction}{0.8}     % max fraction of floats at bottom

% Parameters for TEXT pages (not float pages):
\setcounter{topnumber}{2}
\setcounter{bottomnumber}{2}
\setcounter{totalnumber}{4}             % 2 may work better
\setcounter{dbltopnumber}{2}            % for 2-column pages
\renewcommand{\dbltopfraction}{0.9}     % fit big float above 2-col. text
\renewcommand{\textfraction}{0.07}      % allow minimal text w. figs

% Parameters for FLOAT pages (not text pages):
\renewcommand{\floatpagefraction}{0.7}          % require fuller float pages

% N.B.: floatpagefraction MUST be less than topfraction !!
\renewcommand{\dblfloatpagefraction}{0.7}       % require fuller float pages


%%%%%%%%%%%%%%%%%%%%%%% options for sweave %%%%%%%%%%%%%%%%%%%%%%%%%%%%%%%%%%%%%%%%%%%%


%%%%%%%%%%%%%%%%%%%%%% headers and footers %%%%%%%%%%%%%%%%%%%%%%%%%%%%%%%%%%%%%%%%%%%%
\pagestyle{fancy} 
\renewcommand{\footrulewidth}{\headrulewidth}

%%%%%%%%%%%%%%%%%%%%%%%%% bibliography  %%%%%%%%%%%%%%%%%%%%%%%%%%%%%%%%%%%%%%%%%%%%%%%
\bibliographystyle{plainnat}


%%%%%%%%%%%%%%%%%%%%%%%%%%%%%%%%%%%%%%%%%%%%%%%%%%%%%%%%%%%%%%%%%%%%%%%%%%%%%%%%%%%%%%%
%%%%%%%%%%%%%%%%%%%%%%%%% Now Edit %%%%%%%%%%%%%%%%%%%%%%%%%%%%%%%%%%%%%%%%%%%%%%%%%%%%
%%%%%%%%%%%%%%%%%%%%%%%%%%%%%%%%%%%%%%%%%%%%%%%%%%%%%%%%%%%%%%%%%%%%%%%%%%%%%%%%%%%%%%%
\fancyhead[L]{\em Leonardo Collado Torres}
\fancyhead[R]{\em Ejercicios 4}
\fancyfoot[L]{\em \today}
\fancyfoot[C]{}
\fancyfoot[R]{\em \thepage}


%%%%%%%%%%%%%%%%%%%%%%% opening %%%%%%%%%%%%%%%%%%%%%%%%%%%%%%%%%%%%
\title{Ejercicios 4}
\author{Leonardo Collado Torres}

\usepackage{Sweave}
\begin{document}
\maketitle

\bigskip
Entrega tu c�digo para resolver los siguientes ejercicios v�a la p�gina de cursos. Tu c�digo debe ser portable y tiene que tener un nombre como lcollado.R aunque de preferencia pr�ctica usar \texttt{Sweave} en cuyo caso entrega solo un archivo PDF.
\bigskip

\begin{enumerate}
  
  \item Asume que una poblaci�n est� dividia en dos sobre alguna decisi�n ($p = 1/2$). Toman una muestra aleatoria de tama�o 1000. Cual es la probabilidad de que la muestra aleatoria tendr� 550 o m�s votos a favor de la decisi�n? Responde usando una aproximaci�n normal.
  \bigskip

  \item Para qu� valor de $n$ la distribuci�n $\overline{X}$ se ve aproximadamente como una normal cuando cada $X_i$ es de tipo exponencial (1)\footnote{\texttt{rexp(n,1}}. Corre varias simulaciones para diferentes $n$s hasta que decidas cuando empieza a verse como normal.
  \bigskip
  
  \item Usa el set de datos \texttt{cabinet} del paquete \texttt{UsingR}. Este contiene informaci�n sobre la cantidad de dinero que se ahorrar�n los miembros del comit� de Bush en el 2003 ccuando pasaron una nueva ley sobre impuestos. Esta informaci�n est� almacenada en la variable \texttt{est.tax.savings}. Encuentra un intervalo de confianza de 90\% para la mediana.
  \bigskip
  
  \item El set de datos \texttt{babies} de \texttt{UsingR} tiene las edades de las madres en la variable \texttt{age} y la de los padres en \texttt{dage}. Haz una prueba de significancia de la hip�tesis nula de edades iguales contra una de alternativa de un solo lado donde los pap�s est�n m�s viejos. Reporta el valor $p$ y tu conclusi�n sobre $H_0$.
  \bigskip
  
  \item El set de datos \texttt{pi2000} de \texttt{UsingR} tiene los primeros 2000 d�gitos de $\pi$. Haz una prueba de hip�tesis para ver si los d�gitos aparecen con la misma probabilidad. Usa la prueba de la $\chi^2$.
  \bigskip
  
  \bigskip
\end{enumerate}


\end{document}
