\documentclass[letterpaper,12pt]{article}

%%%%%%%%%%%%%%%%%%%%%%%% Standard Packages %%%%%%%%%%%%%%%%%%%%%%%%%%%%%%%%%%%%%%%%%%
\usepackage{epsfig}
\usepackage{graphicx}
\usepackage{graphics}
\usepackage{amssymb}
\usepackage{amsmath}
\usepackage{mathrsfs}
\usepackage{fancyvrb}
\usepackage{caption}
\usepackage{comment}
\usepackage{fancyhdr}

%%%%%%%%%%%%%%%%%%%%%%%% Adapted from Sweave %%%%%%%%%%%%%%%%%%%%%%%%%%%%%%%%%%%%%%%%%

\DefineVerbatimEnvironment{Rcode}{Verbatim}{fontshape=sl, frame=single, 
  framesep=2mm, fontsize=\small, baselinestretch=.5}

%%%%%%%%%%%%%%%%%%%%%%%%%%%%%%%%%%%%%%%%%%%%%%%%%%%%%%%%%%%%%%%%%%%%%%%%%%%%%%%%%%%%%%

%%%%%%%%%%%%%%%%%%%%%%%% My macros (which of course are borrowed from a million ... %%
\def\argmax{\operatornamewithlimits{arg\,max}}
\def\argmin{\operatornamewithlimits{arg\,min}}


%%%%%%%%%%%%%%%%%%%%%%%% Page and Document Setup %%%%%%%%%%%%%%%%%%%%%%%%%%%%%%%%%%%%%
\addtolength{\oddsidemargin}{-0.875in}
\addtolength{\topmargin}{-0.875in}
\addtolength{\textwidth}{1.75in}
\addtolength{\textheight}{1.75in}

\captionsetup{margin=15pt,font=small,labelfont=bf}

\renewcommand{\topfraction}{0.9}        % max fraction of floats at top
\renewcommand{\bottomfraction}{0.8}     % max fraction of floats at bottom

% Parameters for TEXT pages (not float pages):
\setcounter{topnumber}{2}
\setcounter{bottomnumber}{2}
\setcounter{totalnumber}{4}             % 2 may work better
\setcounter{dbltopnumber}{2}            % for 2-column pages
\renewcommand{\dbltopfraction}{0.9}     % fit big float above 2-col. text
\renewcommand{\textfraction}{0.07}      % allow minimal text w. figs

% Parameters for FLOAT pages (not text pages):
\renewcommand{\floatpagefraction}{0.7}          % require fuller float pages

% N.B.: floatpagefraction MUST be less than topfraction !!
\renewcommand{\dblfloatpagefraction}{0.7}       % require fuller float pages


%%%%%%%%%%%%%%%%%%%%%%% options for sweave %%%%%%%%%%%%%%%%%%%%%%%%%%%%%%%%%%%%%%%%%%%%


%%%%%%%%%%%%%%%%%%%%%% headers and footers %%%%%%%%%%%%%%%%%%%%%%%%%%%%%%%%%%%%%%%%%%%%
\pagestyle{fancy} 
\renewcommand{\footrulewidth}{\headrulewidth}

%%%%%%%%%%%%%%%%%%%%%%%%% bibliography  %%%%%%%%%%%%%%%%%%%%%%%%%%%%%%%%%%%%%%%%%%%%%%%
\bibliographystyle{plainnat}


%%%%%%%%%%%%%%%%%%%%%%%%%%%%%%%%%%%%%%%%%%%%%%%%%%%%%%%%%%%%%%%%%%%%%%%%%%%%%%%%%%%%%%%
%%%%%%%%%%%%%%%%%%%%%%%%% Now Edit %%%%%%%%%%%%%%%%%%%%%%%%%%%%%%%%%%%%%%%%%%%%%%%%%%%%
%%%%%%%%%%%%%%%%%%%%%%%%%%%%%%%%%%%%%%%%%%%%%%%%%%%%%%%%%%%%%%%%%%%%%%%%%%%%%%%%%%%%%%%
\fancyhead[L]{\em Leonardo Collado Torres}
\fancyhead[R]{\em ej1}
\fancyfoot[L]{\em \today}
\fancyfoot[C]{}
\fancyfoot[R]{\em \thepage}


%%%%%%%%%%%%%%%%%%%%%%% opening %%%%%%%%%%%%%%%%%%%%%%%%%%%%%%%%%%%%
\title{Ejercicios 3}
\author{Leonardo Collado Torres}

\usepackage{Sweave}
\begin{document}
\maketitle

\bigskip
Entrega tu c�digo para resolver los siguientes ejercicios v�a la p�gina de cursos. Tu c�digo debe ser portable y tiene que tener un nombre como lcollado.R
\bigskip

\begin{enumerate}
  
  \item Usen el set de datos \texttt{diamond} del paquete \texttt{UsingR}.
  \begin{enumerate}
    \item Hagan un scatterplot de las variables \texttt{carat} y \texttt{price}.
	\item A�adenle la l�nea de regresi�n lineal simple.
	\item Haz una predicci�n del valor de un diamante de un tercio de carat con esta regresi�n.
  \end{enumerate}
  \bigskip

  \item Usen el set de datos \texttt{cancer} del paquete \texttt{UsingR}. 
  \begin{enumerate}
    \item Haz una gr�fica con los boxplots de las variables \texttt{stomach}, \texttt{bronchus}, \texttt{colon}, \texttt{ovary} y \texttt{breast}.
	\item Cual tiene la cola m�s larga?
	\item Cual es la m�s compacta?
	\item Todos los centros son similares?	
  \end{enumerate}
  \bigskip
  
  \item Usa el set de datos \texttt{kid.weights} del paquete \texttt{UsingR}.
  \begin{enumerate}
    \item Explora la relaci�n entre el peso y la edad usando \texttt{lattice}.
    \item Particiona a las edades en intervalos de 0-3, 3-6, 6-9 y 9-12. Se mantiene la misma relaci�n en los grupos?
  \end{enumerate}
  \bigskip
  
  \item Usa el set de datos \texttt{female.inc} del paquete \texttt{UsingR}. Son datos de mujeres en el 2001 en USA.
  \begin{enumerate}
    \item Hay alguna diferencia del ingreso entre las razas? Usa boxplots con \texttt{lattice} para ayudarte a responder la pregunta.
	\item Encuentra el resumen de estad�sticas para cada grupo racial.
  \end{enumerate}
  \bigskip
  
  \bigskip
\end{enumerate}


\end{document}
