\documentclass[letterpaper,12pt]{article}

%%%%%%%%%%%%%%%%%%%%%%%% Standard Packages %%%%%%%%%%%%%%%%%%%%%%%%%%%%%%%%%%%%%%%%%%
\usepackage{epsfig}
\usepackage{graphicx}
\usepackage{graphics}
\usepackage{amssymb}
\usepackage{amsmath}
\usepackage{mathrsfs}
\usepackage{fancyvrb}
\usepackage{caption}
\usepackage{comment}
\usepackage{fancyhdr}

%%%%%%%%%%%%%%%%%%%%%%%% Adapted from Sweave %%%%%%%%%%%%%%%%%%%%%%%%%%%%%%%%%%%%%%%%%

\DefineVerbatimEnvironment{Rcode}{Verbatim}{fontshape=sl, frame=single, 
  framesep=2mm, fontsize=\small, baselinestretch=.5}

%%%%%%%%%%%%%%%%%%%%%%%%%%%%%%%%%%%%%%%%%%%%%%%%%%%%%%%%%%%%%%%%%%%%%%%%%%%%%%%%%%%%%%

%%%%%%%%%%%%%%%%%%%%%%%% My macros (which of course are borrowed from a million ... %%
\def\argmax{\operatornamewithlimits{arg\,max}}
\def\argmin{\operatornamewithlimits{arg\,min}}


%%%%%%%%%%%%%%%%%%%%%%%% Page and Document Setup %%%%%%%%%%%%%%%%%%%%%%%%%%%%%%%%%%%%%
\addtolength{\oddsidemargin}{-0.875in}
\addtolength{\topmargin}{-0.875in}
\addtolength{\textwidth}{1.75in}
\addtolength{\textheight}{1.75in}

\captionsetup{margin=15pt,font=small,labelfont=bf}

\renewcommand{\topfraction}{0.9}        % max fraction of floats at top
\renewcommand{\bottomfraction}{0.8}     % max fraction of floats at bottom

% Parameters for TEXT pages (not float pages):
\setcounter{topnumber}{2}
\setcounter{bottomnumber}{2}
\setcounter{totalnumber}{4}             % 2 may work better
\setcounter{dbltopnumber}{2}            % for 2-column pages
\renewcommand{\dbltopfraction}{0.9}     % fit big float above 2-col. text
\renewcommand{\textfraction}{0.07}      % allow minimal text w. figs

% Parameters for FLOAT pages (not text pages):
\renewcommand{\floatpagefraction}{0.7}          % require fuller float pages

% N.B.: floatpagefraction MUST be less than topfraction !!
\renewcommand{\dblfloatpagefraction}{0.7}       % require fuller float pages


%%%%%%%%%%%%%%%%%%%%%%% options for sweave %%%%%%%%%%%%%%%%%%%%%%%%%%%%%%%%%%%%%%%%%%%%


%%%%%%%%%%%%%%%%%%%%%% headers and footers %%%%%%%%%%%%%%%%%%%%%%%%%%%%%%%%%%%%%%%%%%%%
\pagestyle{fancy} 
\renewcommand{\footrulewidth}{\headrulewidth}

%%%%%%%%%%%%%%%%%%%%%%%%% bibliography  %%%%%%%%%%%%%%%%%%%%%%%%%%%%%%%%%%%%%%%%%%%%%%%
\bibliographystyle{plainnat}


%%%%%%%%%%%%%%%%%%%%%%%%%%%%%%%%%%%%%%%%%%%%%%%%%%%%%%%%%%%%%%%%%%%%%%%%%%%%%%%%%%%%%%%
%%%%%%%%%%%%%%%%%%%%%%%%% Now Edit %%%%%%%%%%%%%%%%%%%%%%%%%%%%%%%%%%%%%%%%%%%%%%%%%%%%
%%%%%%%%%%%%%%%%%%%%%%%%%%%%%%%%%%%%%%%%%%%%%%%%%%%%%%%%%%%%%%%%%%%%%%%%%%%%%%%%%%%%%%%
\fancyhead[L]{\em Leonardo Collado Torres}
\fancyhead[R]{\em ej1}
\fancyfoot[L]{\em \today}
\fancyfoot[C]{}
\fancyfoot[R]{\em \thepage}


%%%%%%%%%%%%%%%%%%%%%%% opening %%%%%%%%%%%%%%%%%%%%%%%%%%%%%%%%%%%%
\title{Ejercicios 1: probando lo visto en la intro}
\author{Leonardo Collado Torres}

\usepackage{/usr/local/lib/R/share/texmf/Sweave}
\begin{document}
\maketitle
\begin{enumerate}
  \item Empiezen por hacer el ejercicio simple de tarea que les puse en la clase.
  \bigskip
  
  \item Abran \texttt{R} y utilicen los datos \texttt{2 5 4 10 8} para:
  \begin{enumerate}
    \item Almacenarlos en un vector de datos x.
	\item Encuentren el cuadrado de cada n�mero.
	\item Substraigan 3 de cada n�mero.
	\item Substraigan 5 de cada n�mero y luego encuentren su ra�z.
  \end{enumerate}
  Hagan esto usando funciones (aprov�chense del reciclaje).
  \bigskip
  
  \item Siendo \texttt{x} \& \texttt{y}:
\begin{Schunk}
\begin{Sinput}
> x <- c(1, 3, 5, 7, 9)
> y <- c(2, 3, 5, 7, 11, 13)
\end{Sinput}
\end{Schunk}
  Sin correr los comandos, cual es el resultado de:
  \begin{enumerate}
    \item x+1
    \item y*2
	\item length(x) y length(y)
	\item x + y (hay reciclaje)
	\item sum(x>5) y sum(x[x>5])
	\item sum(x>5 | x<3)
	\item y[3]
	\item y[-3]
	\item y[x] (Que es NA?)
	\item y[y>=7]
  \end{enumerate}
  \bigskip
  
  \item Usen el conjunto de datos \texttt{islands} que viene en \texttt{R}. Este contiene el tama�o de los bloques de tierra cuya �rea excede las 10 mil millas cuadradas. Encuentren las 7 m�s grandes y las 2 m�s chicas.
  Les recomiendo que usen lo visto en �ndices de vectores y la funci�n \texttt{sort}.
  \bigskip

  \item Qu� imprime el siguiente c�digo?
\begin{Schunk}
\begin{Sinput}
> funej <- function(x) {
+     x <- x^2 + x
+     return(x)
+ }
> x <- 5
> funej(x)
> x
\end{Sinput}
\end{Schunk}
  \bigskip
  
  \item Con los n�meros del 65 al 89, construye una matrix 5x5 llamada \texttt{M}.
  \begin{enumerate}
    \item Encuentra la matrix \texttt{M5} al multiplicar \texttt{M} por el vector 5.
    \item Multiplica \texttt{M5} por \texttt{M} para encontrar \texttt{N}.	Encuentra "manualmente" el valor de la casilla (3,4) de \texttt{N} y luego verifica tu resultado con \texttt{R}.
	\item Encuentra el resultado de la ecuaci�n (1):
	\begin{equation}
N - M(M^TM)^{-1}M^TN
    \end{equation}
  \end{enumerate}
  \bigskip
  
  \item Ahora si viene el ejercicio bueno :D.
  \begin{enumerate}
    \item Almacenen en un \texttt{data frame} el archivo de datos \texttt{fagos\_grandes\_codon.txt}. Usen cualquiera la forma para leer archivos que mas se acomode a los datos o que les guste usar.
	\item Escriban una funci�n para encontrar la suma de cada columna: tienen que usar el ciclo \texttt{for}. Estos resultados los pueden almacenar en un vector; relativ�cenlos con la suma total de las todas columnas para encontrar la frecuencia relativa de cada columna. NOTA: Si quieren hacer un ejercicio m�s avanzado, relativicen las columnas por amino �cido codificado; por ejemplo la columna D\_GAT tendr�a el valor de (D\_GAT) / ( D\_GAT + D\_GAC ).
	\item Encuentren los nombres de las columnas donde la frecuencia relativa\footnote{La que encontraron en el anterior punto} es mayor a 3\%\footnote{Son 6 columnas :)}. Para los que hagan la versi�n avanzada, encuentren el cod�n m�s frecuente para todos los amino �cidos.
  \end{enumerate}
  \bigskip
  
\end{enumerate}


\end{document}
